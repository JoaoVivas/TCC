
% para os exemplos do manual
%\newenvironment{exemplomanual}{
%\vspace{0.5cm}
%\noindent\begin{minipage}{\textwidth}
%\noindent\rule{\textwidth}{0.5pt}
%\vspace{-1cm}
%\begin{flushleft}
%}{
%\end{flushleft}
%\vspace{-0.6cm}
%\noindent\rule{\textwidth}{0.5pt}
%\vspace{0.3cm}
%\end{minipage}
%}

%\newenvironment{exemplomanuallista}{
%\vspace{0.3cm}
%\noindent\begin{minipage}{\textwidth - 0.5cm}
%\noindent\rule{\textwidth}{0.5pt}
%\vspace{-1cm}
%\begin{flushleft}
%}{
%\end{flushleft}
%\vspace{-0.6cm}
%\noindent\rule{\textwidth}{0.5pt}
%\vspace{0.3cm}
%\end{minipage}
%}

% por conta de alguns exemplos
%\usepackage{setspace}

%##################################################


% se vc já defendeu e tem o arquivo escaneado da folha de rosto, 
% descomente e altere o nome do arquivo


%##################################################

% Dados do guia
%\begin{titlepage}
%\pagestyle{empty}
%\renewcommand{\baselinestretch}{1}
%\enlargethispage{1.5cm}
%\input{reitoria}
%\cleardoublepage
%\end{titlepage}

%##################################################

% descomente para habilitar a lista desejada
\listoffigures                             % Lista de Figuras
%\listofilustracoes
%\listofgraficos							   % Lista de Gráficos
\listoftables                              % Lista de Tabelas
%\listofquadros							   % Lista de Quadros
%\listofexemplos
%\listofteoremas
\tableofcontents                           % Sumário



%==============================================================================
% Incluindo bibliografia
%\bibliographystyle{plain}             % estilo para labels em numeros
%\bibliographystyle{alpha}             % estilo para labels em iniciais
\bibliographystyle{abntex2-alf}           % estilo para referências usando ABNT, 
                                       % precisa instalar o abntex para usar!!!

%inclui Referências Bibliográficas
%inclui Referências Bibliográficas
\referencias
\bibliography{referencias}			% arquivo exemplo refbib.bib
%==============================================================================
% Incluindo anexos num1erados com letras maiusculas.
%\apendices

% arquivo exemplo refbib.bib
%==============================================================================
% Incluindo anexos num1erados com letras maiusculas.
%\apendices


%==============================================================================
% Dados da monografia, capa: autor, titulo, banca, etc... - SUBSTITUA DE ACORDO
%==============================================================================


\author{Joao Vivas Cisalpino}
\title{CONTROLE DE VELOCIDADE EM  IMPRESSORAS 3D UTILIZANDO O PRINCÍPIO DE 
FEEDFORWARD PARA OTIMIZAR A VELOCIDADE E QUALIDADE DE IMPRESSÃO.
}
%\subtitle{Exemplo para os Usuários}
%\engtitle{Use of Uflamon Class}
% \engsubtitle{Subtitulo teste}
%\edicao{3$^a$ edição revista, atualizada e ampliada}
\date{2023}
\tipo{Monografia apresentada à Universidade Federal de Lavras,
como parte das exigências Curso de Engenharia Mecânica,
para a obtenção do título de Bacharel.}
% use \orientador ou \orientadora quando for o caso
\orientador{Prof. Dr. Wander Gustavo Rocha Vieira}
%\orientadora{}
% use \coorientador ou \coorientadora quando for o caso
%\coorientadora{Camila Rodrigues Dias} % comente se não tiver coorientador
%\coorientador{}
\local{Lavras - MG}
\bancaum{Ainda não definido}{UFLA}
\bancadois{Ainda não definido}{UFLA?} % comente se sua banca tiver só um professor
\bancatres{Ainda não definido 3}{BELMIS}
%\bancaquatro{Prof. Esp. Carlos Banca Quatro}{IBGPLUS}
\defesa{Data da defesa 2023}
%==============================================================================
%##################################################
% Dados para Ficha catalográfica, gerada pelo sistema da Biblioteca da UFLA
% http://www.biblioteca.ufla.br/FichaCatalografica/
% dados para ficha catalográfica
% Elaboração da Ficha Catalográfica
\preparofichacat{Ficha catalográfica elaborada pela Coordenadoria de Processos Técnicos \\ da Biblioteca Universitária da UFLA}
% primeiro autor - como na primeira linha da ficha catalográfica
\fcautor{Cisalpino, Joao Vivas Cisalpino}
% autores, separados por vírgula - na ficha catalográfica, no formato que
% vem após o título e a barra ("/")
\fcautores{Joao Vivas Cisalpino}
% caso trabalho seja ilustrado (figuras, gráficos, tabelas, etc.), 
% então informar por meio do comando a seguir
% caso não seja ilustrado, basta comentá-lo
\fcilustrado{il.}
% dados da edição para a ficha 
\fcedicao{1$^a$ ed. rev., atual. e ampl.}
% tipo do trabalho (tese, dissertação, etc.), de acordo com sistema
% de geração de ficha catalográfica
\fctipo{Trabalho de conclusão de curso(Graduação)}
% ano da defesa, só precisa informar se for diferente do ano da publicação
% se forem iguais, comente a linha a seguir
\fcdatadefesa{2023}
% preencher aqui com os dados de catalogação gerados pelo sistema
\fccatalogacao{1. TCC}
\fcclasi{808.066}


%\antesfichacat{\noindent Para citar este documento: \\UNIVERSIDADE FEDERAL DE LAVRAS. Biblioteca Universitária. \textbf{Manual de normalização e estrutura de trabalhos acadêmicos: TCC, monografias, dissertações e teses}. 2. ed. rev., atual. e ampl. Lavras, 2015. Disponível em: \url{http://www.biblioteca.ufla.br/wordpress/wpcontent/uploads/bdtd/manual_normalizacao_UFLA.pdf}. Acesso em: data de acesso.}

%\depoisfichacat{\noindent A reprodução e a divulgação total ou parcial deste trabalho são autorizadas, por qualquer meio convencional ou eletrônico, para fins de estudo e pesquisa, desde que citada a fonte.\\
%\newline
%{\small Este documento possui páginas em branco para facilitar a impressão frente-e-verso.}}


% \section{Design centrado no usuário}

% \begin{itemize}

% \item observação: uma das principais técnicas críticas é observar os possíveis usuários e o seu ambiente, independente de onde o produto será utilizado. As pesquisas de design integram essa etapa, ajudando a determinar as necessidades dos usuários;

% \item geração de ideias: após o levantamento de requisitos, a segunda etapa consiste em gerar potenciais soluções;

% \item prototipação;

% \item testes.

% \end{itemize}

% \subsection{Usabilidade}



% \subsection{Design de interação}

% \section{Experiência do usuário}

% \section{Desenvolvimento ágil}

% \begin{enumerate}

% \item nossa maior prioridade é satisfazer o cliente através de entrega prematura (antecipada) e contínua de software valioso;

% \item necessidades de mudança são bem vindas, mesmo no final do desenvolvimento. Os processos Ágeis utilizam a mudança em favor da vantagem competitiva para o cliente;

% \end{enumerate}

%  Atualmente, existe a Agile Aliance\footnote{https://www.agilealliance.org}, uma organização que surgiu após a declaração dos valores e princípios do Manifesto ágil, comprometida com os avanços dos princípios ágeis. 

% % \begin{figure}[H]
% % \centering
% % \includegraphics[width=1\textwidth]{images/ciclos_scrum.jpg}
% % \caption{Ciclo do Scrum. }
% % \label{fig:cicloscrum}
% % {\small Fonte: http://www.desenvolvimentoagil.com.br/scrum. Acessado em: 01/02/2017} %Fonte da imagem
% % \end{figure}


% \section{Análise e especificação de requisitos}



% O wireframe, como representa a Figura \ref{fig:wireframe} é uma representação de baixa fidelidade de uma interface. Foi reproduzido digitalmente.

% \begin{figure}[H]
% \centering
% \includegraphics[width=1\textwidth]{images/wireframe.png}
% \caption{Wireframe inicial}
% \label{fig:wireframe}
% \end{figure}



% \subsubsection{MarvelApp}
%  O MarvelApp\footnote{https://marvelapp.com} permite a criação de protótipos de alta fidelidade de maneira rápida, eficiente e portátil para testes realizados em smartphones. A partir de imagens e mockups, é possível transformá-los em protótipos para qualquer tipo de dispositivo, sem necessidade de codificação.

% \subsubsection{Axure}
% Soluções podem ser prototipadas e validadas pelas pessoas que melhor compreendem seus negócios, produtos e clientes. O Axure\footnote{https://www.axure.com} permite a criação de flowcharts, wireframes, mockups, personas e quadro de ideias. Além disso, permite uma maior interação juntamente com a entrada de dados pelo usuário.



% \begin{table}[H]
% \centering
% \caption{Severidades}
% \label{tab:severidade}
% \begin{tabular}{c|c|c}
%       \hline
%        \rowcolor[gray]{.9}
%       \bf Severidade  & \bf Frequência & \bf Porcentagem  \\
%       \hline
%       \hline
% 0 (Não acredito que seja um problema) & 0          & 0\%         \\
% 1 (Cosmético)                        & 12         & 14,16\%     \\
% 2 (Simples)                          & 71         & 83,78\%     \\
% 3 (Sério)                           & 35         & 41,3\%      \\
% 4 (Catastrófico)                     & 0          & 0\%         \\

% \hline
% \hline
% \bf TOTAL   & \bf 118        & \bf 100\%   \\
%  \hline 
% \end{tabular}
% \end{table}


% \begin{table}[H]
% \centering
% \caption{ Violações categorizadas por Heurísticas}
% \label{tab:heuristicasAfetadas}
% \begin{tabular}{c|c|c}
%       \hline
%       \rowcolor[gray]{.9}
%  \bf Heurística   &  \bf Frequência & \bf Porcentagem  \\
%       \hline
%       \hline
% H1. Visibilidade do status do sistema
% & 10  & 7\%\\

% H2. Correspondência entre o sistema e o mundo real 
% & 15 & 10\% \\  

% H3. Controle do usuário e liberdade                     
% & 9  & 6\%     \\

% H4. Consistência e padrões                          
% & 29  & 19\%      \\

% H5. Prevenção de erros                        
% &15  & 10\%      \\

% H6. Reconhecimento em vez de recordação                 
% &3  &2\%      \\

% H7. Flexibilidade e eficiência de utilização           
% &5  &3\%      \\

% H8. Estética e design minimalista                       
% & 21  & 14\%      \\

% H9. Ajude os usuários a reconhecer, diagnosticar e resolver erros                
% & 15  &10\%      \\

% H10. Ajuda e documentação                       
% &15  &10\%      \\


% \hline
% \hline
% \bf TOTAL   & \bf 137        & \bf 100\%   \\
%  \hline 
% \end{tabular}
% \end{table}







