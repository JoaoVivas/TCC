

% Utilizacao de pacotes
\usepackage[T1]{fontenc}         % usa fontes postscript com acentos
\usepackage[brazil]{babel}       % hifenização e títulos em português do Brasil
\usepackage[utf8]{inputenc}     % permite edição direta com acentos
\usepackage{amsmath}             % pacote da AMS para Matemática Avançada
\usepackage{amssymb}             % símbolos extras da AMS
\usepackage{latexsym}            % símbolos extras do LaTeX
\usepackage{graphicx}            % para inserção de gráficos
\usepackage{listings}            % para inserção de código
\usepackage{fancyvrb}            % para inserção de saídas de comandos
%\usepackage{enumerate}           % para personalizar lista enumeradas 
											%(incluso na classe)
\usepackage{longtable}           % para tambelas muito grandes NOVO!!!!
\usepackage{pgfplots}
\usepackage{colortbl} % cores em tabelas
\newcolumntype{Z}{|>{\columncolor[gray]{0.9}}l|} %cor cinza em células
%\usepackage{array} % já incluso na classe
\newcolumntype{L}[1]{>{\raggedright\let\newline\\\arraybackslash\hspace{0pt}}m{#1}}
\newcolumntype{C}[1]{>{\centering\let\newline\\\arraybackslash\hspace{0pt}}m{#1}}
\newcolumntype{R}[1]{>{\raggedleft\let\newline\\\arraybackslash\hspace{0pt}}m{#1}}
\usepackage{multirow} % para juntar duas linhas em uma só

\usepackage{multicol} % para uso de várias colunas

% cores para os links cruzados
\usepackage{color}
\definecolor{rltred}{rgb}{0.2,0,0}
\definecolor{rltgreen}{rgb}{0,0.2,0}
\definecolor{rltblue}{rgb}{0,0,0.2}

\usepackage[colorlinks=true,
            urlcolor=black,       % \href{...}{...} external (URL)
            filecolor=black,     % \href{...} local file
            linkcolor=black,       % \ref{...} and \pageref{...}
            citecolor=black,
            pdftitle={Cisalpino,Joao},
          pdfauthor={Joao Vivas Cisalpino},
          pdfsubject={},
          pdfkeywords={Comunicação Científica. 2. Pesquisa . 3. Pesquisa Científica. 
 					 4. Redação. 5. Monografia.}%
]{hyperref} % para referência cruzadas
%\usepackage{hyperref}            % para referência cruzadas
\usepackage{subfigure}           % figuras dentro de figuras
\usepackage{caption}            % remodelando o formato dos títulos de 
                                 % tabelas e figuras

% configuração padrão do listings   
\lstset{
   language=Java,
   extendedchars=true,
   tabsize=3,
   basicstyle=\footnotesize\ttfamily,
   stringstyle=\em,
   showstringspaces=false 
}

% para referências de acordo com a ABNT
% precisa instalar o abntex2 antes!!!
% http://abntex.codigolivre.org.br/
% comente se pretende usar outro padrão

%abnt-emphasize=bf coloca o título das bibliografias em negrito
%abnt-thesis-year=both
\usepackage[alf,abnt-etal-cite=3,abnt-etal-list=3,abnt-url-package=url,abnt-emphasize=bf,abnt-etal-text=emph]{abntex2cite}

% evite usar o hyperref com abntex, pode dar caca em urls... no linha anterior, informo
% para incluir urls usando o pacote url e não o hyperref
%
% caso queira o hyperref com abntex, comente a linha anterior e descomente a seguinte
%\usepackage[alf,abnt-etal-cite=3,abnt-etal-list=0,abnt-etal-text=emph]{abntex2cite}
%
% caso vc ainda use a versão anterior da abntex, comente a linha incluindo o abntex2cite
% e descomente a próxima linha 
%\usepackage[alf,abnt-etal-cite=3,abnt-etal-list=0,abnt-etal-text=emph]{abntcite}


% redefinindo formatação de títulos de tabelas e figuras


%==============================================================================
% para os fãs do Word, descomente as linhas abaixo
%\sloppy %mais espaço entre as linhas
%\usepackage{identfirst} %identando-se a primeira linha de cada seção
%\noindentfirst % Tire o comentário para manter o padrão do LaTeX.

%==============================================================================
% definido comandos na monografia - não é necessário na sua monografia 
% apenas para exemplificar a definição de novos comandos
\newcommand{\defs}[1]{\textsl{#1}}

\graphicspath{{./images}}
% Especificando hifenizações que por ventura LaTeX não saiba fazer
% Por padrão 99,9% dos termos em português devem ser hifenizados corretamente.
\hyphenation{hardware software Li-nux am-bien-te diag-nos-ti-car coor-de-na-ção 
FAE-PE Recovery TelEduc Williams UFLA}
