\chapter{Descrição geral do local do estágio: Progolden Soluções Tecnológicas}

\section{Histórico}

A ProGolden\footnote{http://www.progolden.com.br} é uma startup fundada em 2013 que nasceu a partir da sinergia de conhecimentos e esforços de especialistas, mestres e doutores nas áreas de Gestão Estratégica, Governança, Gestão da Inovação, Sistemas de Informação e Ciência da Computação da Universidade Federal de Lavras (UFLA) e está localizada no Campus Histórico da UFLA.
%TODO As áreas de conhecimento precisam ter iniciais em maiúsculas



\section{Área de atuação}

O principal foco da empresa é fornecer soluções tecnológicas inovadoras com alta qualidade, que agregam valor e diferencial aos seus clientes, buscando em suas ações priorizar e se posicionar continuamente a favor de seus clientes por meio do oferecimento de seus produtos e serviços. 

Devido ao modelo de negócio deste produto pioneiro, que se apoia no conceito de software como serviço (SaaS), a ProGolden oferece seus produtos a baixo custo permitindo que organizações de diferentes portes, sejam pequenas, médias ou grandes, e de diferentes setores, sejam públicas, estatais ou privadas, possam escolhê-los a valores mensais acessíveis, sem custos de instalação ou adesão. 

A Progolden tem como principal produto o Software PrêmioIdeia\footnote{http://www.premioideia.com},
%TODO Colocar menção, mesmo que em nota de rodapé(done)
uma plataforma de gestão de ideias e de inovação, que possibilita a organizações públicas e privadas o desenvolvimento de inovações de alto impacto a um baixo custo, com base em uma metodologia motivada pela gameficação e que utiliza participação e colaboração social. Está disponível na web e em aplicativos para Android e iOS.

\section{Projeto}

O estagiário exerceu, em seu tempo majoritário, atividades no projeto de desenvolvimento de um software para planejamento estratégico. O planejamento estratégico é o processo de analisar uma organização sob vários ângulos, definindo os rumos por meio de um direcionamento que possa ser monitorado nas suas ações concretas \cite{planejamentoEstrategico}. Além disso, o estagiário teve como funções e responsabilidade, junto a equipe: a análise de requisitos, especificação de sistemas e design centrado no usuário. A equipe do projeto foi composta por 6 desenvolvedores, 2 analistas de sistemas e 4 testadores.


O estagiário contribuiu com o projeto por meio de tomadas de decisão, exercendo tarefas de especificação do sistema e design de interação e usabilidade, além disso, auxiliou na correção de problemas encontrados na implementação do software.

%TODO todos os termos em inglês em itálico

Primeiramente, o escopo inicial do sistema foi definido, e a partir dele foi desenvolvido um MVP (Produto Viável Mínimo, do inglês \textit{Minimum Viable Product}) que corresponde a um produto que atende os requisitos iniciais do projeto. O projeto tem duração de um ano, com entregas incrementais a cada três meses, aproximadamente. Foram realizadas reuniões periódicas com o cliente, a fim de validar as funcionalidades desenvolvidas. Atualmente, o software está em fase final de correções, validações e testes e, em breve, será implantado pelo cliente.





