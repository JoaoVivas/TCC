\chapter{CONCLUSÃO}
Conclusão

Nesta monografia, foi abordada a geração de comandos em impressoras 3D utilizando o método de manufatura aditiva conhecido como \textit{Fused Deposition Modeling} (FDM). O objetivo principal do trabalho foi investigar e desenvolver metodologias para possibilitar velocidades de movimentação maiores e garantir a precisão dimensional das peças produzidas.

No referencial teórico (Capítulo 2), foram apresentados os conceitos fundamentais da manufatura aditiva e do FDM, incluindo o uso de algoritmos de controle, análise de trajetórias e modelagem dinâmica do sistema.

A metodologia utilizada (Capítulo 3) envolveu o uso do \textit{software} Matlab, em particular a função fmincon, para a geração de comandos. Foram descritas as etapas para a leitura do \textit{Gcode}, a geração de comandos base, incluindo as curvas trapezoidais de velocidade, a interpolação. Além disso, foram descritas as etapas de implementação do método desenvolvido neste trabalho, incluindo a modelagem dinâmica necessária, as funções de suporte como a integração implícita não linear e o uso da função FMINCON do Matlab.

No Capítulo 4, foram apresentados os resultados e discussões. Correlações entre as variáveis de entrada foram analisadas, bem como as influências do modelo dinâmico nas simulações e seus impactos na peformace computacional. As dificuldades encontradas foram discutidas de maneira explicar a abordagem trilhada pelo presente trabalho, bem como discussões referentes a possiveis maneiras de lidar com os mesmos. Por fim, foi discutido abordagens para trabalhos futuros, como a implementação de uma modelagem do sistema dinâmico para regiões específicas da impressora.

Com base nos resultados obtidos e nas discussões realizadas, conclui-se que a geração de comandos é um aspecto fundamental para a qualidade e eficiência da manufatura aditiva por FDM. As metodologias e técnicas abordadas neste trabalho fornecem uma base sólida para a melhoria contínua do processo, possibilitando um avanço nas tecnicas disponíveis para serem exploradas com objetivo de permitir uma produção de peças com maior precisão dimensional e redução de tempo de fabricação.

Em suma, este estudo representa um passo importante para o aprimoramento da manufatura aditiva por FDM, fornecendo subsídios teóricos e metodológicos para a geração de comandos mais eficientes e precisos. Espera-se que as contribuições deste trabalho incentivem pesquisas adicionais e promovam avanços na área, impulsionando a utilização da manufatura aditiva em diversos setores industriais.
