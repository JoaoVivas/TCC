\chapter{CONCLUSÃO}
Este estudo investigou a geração de comandos em impressoras 3D utilizando o método de \textit{Fused Deposition Modeling} (FDM), com o objetivo principal de desenvolver metodologias que permitam velocidades de movimentação mais rápidas e garantam a precisão dimensional das peças produzidas. Através de um referencial teórico sólido e uma metodologia robusta, este trabalho abordou propostas relevantes sobre o controle de trajetória em impressoras 3D.

A análise de sensibilidade realizada nos diversos casos de estudo destacou como variações em parâmetros específicos, como frequência natural, coeficiente de amortecimento, aceleração, resolução de malha de tempo e velocidade desejada, impactam significativamente o comportamento do sistema simulado. Foi evidenciado que, apesar da robustez e adaptabilidade do sistema a uma ampla gama de condições, existem limites críticos nos quais a eficácia do sistema começa a diminuir. 

Os resultados obtidos demonstraram a eficiência das técnicas e metodologias empregadas, fornecendo uma base sólida para a melhoria contínua do processo de manufatura aditiva por FDM. Este estudo representa um passo importante no aprimoramento da manufatura aditiva, oferecendo subsídios teóricos e metodológicos para a geração de comandos mais eficientes e precisos.

Em termos de aplicações práticas, o trabalho fornece importantes contribuições para a produção de peças com maior precisão dimensional e redução do tempo de fabricação, sendo um incentivo para pesquisas adicionais e avanços na área.

Por fim, este trabalho não apenas alcançou seus objetivos propostos, mas também abriu caminho para futuras investigações e melhorias na área de controle de trajetória para impressoras 3D. Espera-se que as contribuições deste estudo inspirem novas pesquisas e inovações, contribuindo para o avanço da manufatura aditiva como um todo.
