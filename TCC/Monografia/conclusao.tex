\chapter{CONCLUSÃO}
Esta monografia apresentou um método de controle de trajetória para sistemas de impressão 3D, focado em aprimorar a precisão do posicionamento da ferramenta. A implementação do método demonstrou uma diminuição notável no desvio do caminho da ponta quando comparado ao caminho simulado sem controle de trajetória.

Os resultados obtidos por meio de simulações confirmaram a capacidade do controle em atenuar as complexidades dinâmicas do sistema. A simulação de referência, juntamente com as simulações de parâmetros variados, validou a eficácia do algoritmo de controle sob diferentes condições operacionais.

A análise dos parâmetros do sistema, incluindo frequência natural, coeficiente de amortecimento, aceleração de entrada, passo de tempo e velocidade desejada, permitiu quantificar sua influência na resposta do controle.

A pesquisa identificou a importância de ajustar o passo de tempo de acordo com a frequência natural para uma representação mais fiel da resposta do sistema e indicou a execução do controle de trajetória em etapas de diferentes passos de tempo como uma abordagem efetiva para aprimorar o processo de controle.

Em síntese, as contribuições deste trabalho para o campo do controle de trajetória são evidenciadas pela melhoria na precisão da impressão 3D dada pelas simulações e contribui para a exploração de métodos iterativos e de programação linear, estabelecendo um ponto de partida para futuros avanços na otimização de sistemas de controle no contexto de impressoras 3D.