\chapter{metodologia}

\section{Modelagem dinâmica de uma impressora 3D}
Para a modelagem dinâmica dos eixos X e Y da impressora 3D, 
é considerado que os eixos são completamente independentes, 
a flexibilidade da correia é aproximada utilizando um conjunto 
mola amortecedor e a transmissão de movimento e torque dos 
motores é considerada como ideal e não será abordada.
Assim duas posições de estudo surgem para cada eixo, uma delas 
representa a posição ideal e desejada pelo usuário (X1) e a 
segunda é a posição real considerando as forças inerciais e a 
flexibilidade introduzida pela correia (X2) como na Figura 4.

\section{Espaço de estados}
Organizando as equações de movimento dos eixos x e y da 
impressora em espaço de estados em forma matricial, sendo 
Dx, Vx, Ax, Dy, Vy, Ay as variáveis referentes à posição,
velocidade e aceleração das massas x e y e as variáveis Dxb, 
Vxb, Axb, Dyb, Vyb, Ayb referentes a base, relacionadas 
relacionadas pela equação:

equação

Na função não linear as igualdades são calculadas em cada ponto, 
sendo os de índice "i" o ponto atual e "n" o próximo ponto.

\section{Função fmincon}
Como o modelo matemático a ser otimizado é multivariável e 
possuindo restrições não-lineares, a função FMINCON do ambiente 
do MATLAB é utilizada para otimizar as variações de velocidade 
de forma a diminuir o erro de trajetória associado às 
flexibilidades do sistema que causam perturbações e vibrações 
indesejadas.
É uma função baseada em gradientes que busca por todos os 
mínimos locais de uma região que satisfaz outras restrições 
estipuladas (ALBAGHDADI et al. 2021).
Ela utiliza um conjunto de restrições superiores e inferiores 
para cada ponto e otimiza a função considerando as restrições 
estabelecidas pela função não linear, utilizando as equações de 
movimento para encontrar a resolução da EDO de maneira e 
otimizar os parâmetros utilizando o algoritmo sqp, com o 
objetivo de minimizar a seguinte função:

eq

Sendo Dmx e Dmy os vetores com os pontos de deslocamento 
calculados e Dmex e Dmey os vetores de deslocamento desejados, 
ou seja, a trajetória requerida para a impressão.


% \section{Design centrado no usuário}

% \begin{itemize}

% \item observação: uma das principais técnicas críticas é observar os possíveis usuários e o seu ambiente, independente de onde o produto será utilizado. As pesquisas de design integram essa etapa, ajudando a determinar as necessidades dos usuários;

% \item geração de ideias: após o levantamento de requisitos, a segunda etapa consiste em gerar potenciais soluções;

% \item prototipação;

% \item testes.

% \end{itemize}

% \subsection{Usabilidade}



% \subsection{Design de interação}

% \section{Experiência do usuário}

% \section{Desenvolvimento ágil}

% \begin{enumerate}

% \item nossa maior prioridade é satisfazer o cliente através de entrega prematura (antecipada) e contínua de software valioso;

% \item necessidades de mudança são bem vindas, mesmo no final do desenvolvimento. Os processos Ágeis utilizam a mudança em favor da vantagem competitiva para o cliente;

% \end{enumerate}

%  Atualmente, existe a Agile Aliance\footnote{https://www.agilealliance.org}, uma organização que surgiu após a declaração dos valores e princípios do Manifesto ágil, comprometida com os avanços dos princípios ágeis. 

% % \begin{figure}[H]
% % \centering
% % \includegraphics[width=1\textwidth]{images/ciclos_scrum.jpg}
% % \caption{Ciclo do Scrum. }
% % \label{fig:cicloscrum}
% % {\small Fonte: http://www.desenvolvimentoagil.com.br/scrum. Acessado em: 01/02/2017} %Fonte da imagem
% % \end{figure}


% \section{Análise e especificação de requisitos}
