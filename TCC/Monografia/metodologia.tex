\chapter{METODOLOGIA}

\section{Materiais}

\section{Dados base e cofigurações}

aceleração máxima
desvio de junção
delta tempo
kx
ky
bx
by
mx
my


\section{Modelagem dinâmica de uma impressora 3D}
Para a modelagem dinâmica dos eixos X e Y da impressora 3D, 
é considerado que os eixos são completamente independentes, 
a flexibilidade da correia é aproximada utilizando um conjunto 
mola amortecedor e a transmissão de movimento e torque dos 
motores é considerada como ideal e não será abordada.
Assim duas posições de estudo surgem para cada eixo, uma delas 
representa a posição ideal e desejada pelo usuário (X1) e a 
segunda é a posição real considerando as forças inerciais e a 
flexibilidade introduzida pela correia (X2) como na Figura 4.

\subsection{Espaço de estados}
Organizando as equações de movimento dos eixos x e y da 
impressora em espaço de estados em forma matricial, sendo 
Dx, Vx, Ax, Dy, Vy, Ay as variáveis referentes à posição,
velocidade e aceleração das massas x e y e as variáveis Dxb, 
Vxb, Axb, Dyb, Vyb, Ayb referentes a base, relacionadas 
relacionadas pela equação:

equação

Na função não linear as igualdades são calculadas em cada ponto, 
sendo os de índice "i" o ponto atual e "n" o próximo ponto.

\section{Geração de Comando}

\subsection{Leitura Gcode}

Foi considerado no mapeamento do Gcode apenas comandos G1, extraindo
as informações dos eixos X, Y e do *feedrate* (F). Com base nesses valores
uma matriz 3 por n é criada, n sendo o número de comandos lidos.

\subsection{Curva rapezoidal de velocidade}
Para a construção da curva de velocidade trapezoidal a partir da matriz de posições
e *feedrate* é necessário o calculo das direções dos movimentos a serem realizados.

As direções são representadas por vetores unitários calculados a partir
do vetor deslocamento dividido pelo mesmo vetor normalizado. Sendo o vetor deslocamento
obtido pela diferença entre dois vetores de posição sequenciais.

\begin{equation}
    \label{eq:dir_eq}
    dir_{vector} = \frac{v}{norm(v)}
\end{equation}

Essas direções são utilizadas para o cálculo das velocidades pelo *look ahead*,
considerando os valores de desvio de junta como  0.1 e aceleração maxima de $5000 mm/s^2$

\begin{equation}
    \label{eq:alpha}
    \alpha = 2*asin\left(\frac{norm(dir_1+dir_2)}{2}\right)
\end{equation}

\begin{equation}
    \label{eq:div}
    div = \frac{1-sin\left(\frac{\alpha}{2}\right)}{sin(\frac{\alpha}{2})}
\end{equation}

\begin{equation}
    \label{eq:R_jun}
    R = \frac{jun_disv}{div}
\end{equation}

\begin{equation}
    \label{eq:v_jun}
    v_jun = \sqrt{acc_max*R}
\end{equation}


Calculamos então a velocidade de junção baseada no angulo das direções do movimento e
nos parâmetros da aceleração máxima e desvio máximo de junção.

A partir dessa matriz de posições e, agora também velocidades,
utilizamos a função responsável por gerar a curva trapezoidal de velocidades.

Essa função separa o deslocamento total do movimento em 3 ou 2 fases de aceleração constante.
É utilizado a equação x para o cálculo da velocidade pico

\begin{equation}
    \label{eq:v_p}
    v_p = \sqrt{\frac{(v_i^2+v_f^2)}{2}+acc*des_{tot}}
\end{equation}

\begin{equation}
    \label{eq:des_seg_acc}
    des_{segment} = \frac{(v_f^2-v_i^2)}{(2*acc_{segment})}
\end{equation}

\begin{equation}
    \label{eq:des_seg_no_acc}
    des_{middle} = des_{total}-(des_{up}+des_{down})
\end{equation}

\begin{equation}
    \label{eq:dt_seg_acc}
    dt_{segment} = \frac{(v_f-v_i)}{acc_{segment}}
\end{equation}

\begin{equation}
    \label{eq:dt_seg_no_acc}
    dt_d = \frac{des_d}{v_d}
\end{equation}

\begin{equation}
    \label{eq:delta_vel}
    \Delta vel = v_f-v_i
\end{equation}

\begin{equation}
    \label{eq:acumulator_function}
    \begin{split}
        v_{0} &= init_{value} + v_{0} \\
        v_k &= v_k+v_{k-1}
    \end{split}
\end{equation}

\begin{equation}
    \label{eq:N_steps}
    N_{steps} = \lceil\frac{\Delta t_i}{\Delta t_{step_{size}}}-1\rceil
\end{equation}

\begin{equation}
    \label{eq:dt_interpol_last_step}
    \Delta t_{last_{step}}= \Delta t_i - \Delta t_{step_{size}}*N_{steps} 
\end{equation}

\begin{equation}
    \label{eq:delta_des_interpol}
    \Delta des_i = \Delta v_i*\Delta t_i+ \frac{acc_{segment}*\Delta t_i^2}{2} 
\end{equation}


**Matrizes de exemplo**

A partir da comparação da velocidade pico com a velocidade desejada pelo *feedrate*.
Caso a velocidade de pico for maior do que a velocidade desejada, temos 3 fases de deslocamento
que são calculadas pelas equações x,x e x
Caso a velocidade de pico seja igual ou menor do que a velocidade desejada, teremos 2 fases de deslocamento
que são calculadas a partir das equações x e x.

Como resultado da função, obtemos uma nova matriz esta contendo informações
sobre o a variação da posição, do tempo, da velocidade e sobre a aceleração e direção de deslocamento no ponto.

A partir dessa matriz, utilizamos a função de interpolação para dividir cada intervalo dessa matriz em intervalos menores
baseados em uma configuração de passo de interpolação, no caso baseado em passos de tempo.
Assim criando-se uma nova matriz dos dados interpolados.

A partir dos vetores de direção e da função acumuladora que se consite em acumular os valores de um vetor.
Obtemos uma matriz de posições, velocidades e tempo.

\subsection{Interpolação}

\section{Runge Kutta}



% s(:,1) = s0;
    
%     A = A_model;
%     B = B_model;
%     N = length(t)-1;
%     for i=1:N
%         dt(i) = t(i+1)-t(i);
%         uhalf = u_t_interpolator(u(:,i),u(:,i+1),dt(i),dt(i)/2);

%         k1(:,i) = dynamic_model(s(:,i),u(:,i),A,B);
%         k2(:,i) = dynamic_model(s(:,i)+k1(i)*dt(i)/2,uhalf,A,B);
%         k3(:,i) = dynamic_model(s(:,i)+k2(i)*dt(i)/2,uhalf,A,B);
%         k4(:,i) = dynamic_model(s(:,i)+k3(i)*dt(i),u(:,i+1),A,B);

%         avg_dot = (k1(:,i) + 2*k2(:,i) + 2*k3(:,i) + k4(:,i))/6;
%         s(:,i+1) = s(:,i) + dt(i)*avg_dot;
%     end


Para o calculo da estimativa da respota do sistema, utilizamos a função Runge Kutta
Primeiramente calculamos os valores de k1,k2,k3 e k4, calculamos a média da derivada do vetor de variaveis
e por fim o calculo do vetor x.

\section{Função fmincon}
Como o modelo matemático a ser otimizado é multivariável e 
possuindo restrições não-lineares, a função FMINCON do ambiente 
do MATLAB é utilizada para otimizar as variações de velocidade 
de forma a diminuir o erro de trajetória associado às 
flexibilidades do sistema que causam perturbações e vibrações 
indesejadas.
É uma função baseada em gradientes que busca por todos os 
mínimos locais de uma região que satisfaz outras restrições 
estipuladas \cite{albaghdadi21}.
Ela utiliza um conjunto de restrições superiores e inferiores 
para cada ponto e otimiza a função considerando as restrições 
estabelecidas pela função não linear, utilizando as equações de 
movimento para encontrar a resolução da EDO de maneira e 
otimizar os parâmetros utilizando o algoritmo sqp, com o 
objetivo de minimizar a seguinte função:

eq

Sendo Dmx e Dmy os vetores com os pontos de deslocamento 
calculados e Dmex e Dmey os vetores de deslocamento desejados, 
ou seja, a trajetória requerida para a impressão.

\subsection{Restrições lineares}
As matrizes de upper bound e lowerbound foram construidas com as seguites considerações:


% N = length(x(5,:));
% base = ones(1,N);

% lb(1,:) = base*min_x;
% lb(2,:) = base*min_y;
% lb(3,:) = base*min_x; % des_ux
% lb(4,:) = base*min_y; % vel_ux
% lb(5,:) = base*-0.0002;

% ub(1,:) = base*max_x;
% ub(2,:) = base*max_y;
% ub(3,:) = base*max_x; 
% ub(4,:) = base*max_y; 
% ub(5,:) = (x(5,:)+0.0002)*1.2;

**considerações**

\subsection{Restrições não lineares}
Como restrições não lineares foi implementada uma função que compara
o resultado do Runge Kutta com o vetor posição da fmincon.


% t = x(5,:);

% des_x = x(1,:);
% des_y = x(2,:);
% vel_x(1) = 0;
% vel_y(1) = 0;
% acc_x(1) = 0;
% acc_y(1) = 0;

% des_xb = x(3,:);
% des_yb = x(4,:);
% vel_xb(1) = 0;
% vel_yb(1) = 0;
% acc_xb(1) = 0;
% acc_yb(1) = 0;

% %dv = des/dt; da = dv/dt ; 
% %des = vi*t+at^2/2; a = 2*(des/t-vi)/t
% % Vf^2-Vi^2 = 2ades

% ceq=[des_x(1),des_y(1),t(1), des_xb(1), des_yb(1)];    
% c=[];

% for i = 1 : (length(t)-1)
%     dt = t(i+1)-t(i);
    
%     delta_x = (des_x(i+1)-des_x(i));
%     delta_y = (des_y(i+1)-des_y(i));
    
%     acc_x(i+1) = 2*((delta_x/dt)-vel_x(i))/dt; 
%     acc_y(i+1) = 2*((delta_y/dt)-vel_y(i))/dt;
%     vel_x(i+1) = vel_x(i)+acc_x(i+1)*dt;
%     vel_y(i+1) = vel_y(i)+acc_y(i+1)*dt;

%     delta_xb = (des_xb(i+1)-des_xb(i));
%     delta_yb = (des_yb(i+1)-des_yb(i));
    
%     acc_xb(i+1) = 2*((delta_xb/dt)-vel_xb(i))/dt;
%     acc_yb(i+1) = 2*((delta_yb/dt)-vel_yb(i))/dt;
%     vel_xb(i+1) = vel_xb(i)+acc_xb(i+1)*dt;
%     vel_yb(i+1) = vel_yb(i)+acc_yb(i+1)*dt;

%     AccelXb = abs((vel_xb(i+1)-vel_xb(i))/dt);
%     AccelYb = abs((vel_yb(i+1)-vel_yb(i))/dt);
    
%     c = [c max_acc-AccelXb max_acc-AccelYb];
    
%     x_i = [des_x(i);vel_x(i);des_y(i);vel_y(i)];
%     u_i = [des_xb(i);vel_xb(i);des_yb(i);vel_yb(i)];
    
%     f_i = A_model*x_i+B_model*u_i;
    
%     x_n = [des_x(i+1);vel_x(i+1);des_y(i+1);vel_y(i+1)];
%     u_n = [des_xb(i+1);vel_xb(i+1);des_yb(i+1);vel_yb(i+1)];
    
%     f_n = A_model*x_n+B_model*u_n;
    
%     x_c =(x_i+x_n)/2+dt*(f_i-f_n)/8;
%     u_c = (u_i+u_n)/2;
    
%     x_ca=-3*(x_i-x_n)/(2*dt)-(f_i+f_n)/4;
    
%     f_c = A_model*x_c+B_model*u_c;
    
%     delta = f_c-x_ca;
%     ceq=[ceq delta'];
% end


\subsection{Cofiguração da Fmincon}
Foi utilizado as seguintes configurações da função fmincon:

**configurações**

optimoptions(@fmincon, 'TolFun', 0.000000001, 'MaxIter', 100000, ...
                       'MaxFunEvals', 700000, 'Display', 'iter', ...
                       'DiffMinChange', 0.0001, 'Algorithm', 'interior-point', 'StepTolerance', 1e-12);

adas


Equações

In line equations $x = 3$

Full line equations

$$y = x+3$$

Referenceble equation

\begin{equation}
\label{eq:teste}
x+1 = 6+7
\end{equation}

\begin{table}
    \begin{center}
    \caption{tabela de teste}
    \label{tab:teste}
    \begin{tabular}{c c c}
        teste & teste2 & teste3 \\ \hline
        1 & 2 & 3 \\
        4 & 5 & x+1 \\ \hline
    \end{tabular}
    \end{center}
\end{table}
                       
