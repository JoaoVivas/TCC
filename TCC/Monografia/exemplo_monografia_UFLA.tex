\documentclass{uflamon}          % classe base para a monografia

%==============================================================================
% Utilizacao de pacotes
\usepackage[T1]{fontenc}         % usa fontes postscript com acentos
\usepackage[brazil]{babel}       % hifenização e títulos em português do Brasil
\usepackage[utf8]{inputenc}     % permite edição direta com acentos
\usepackage{amsmath}             % pacote da AMS para Matemática Avançada
\usepackage{amssymb}             % símbolos extras da AMS
\usepackage{latexsym}            % símbolos extras do LaTeX
\usepackage{graphicx}            % para inserção de gráficos
\usepackage{listings}            % para inserção de código
\usepackage{fancyvrb}            % para inserção de saídas de comandos
%\usepackage{enumerate}           % para personalizar lista enumeradas 
											%(incluso na classe)
\usepackage{longtable}           % para tambelas muito grandes NOVO!!!!

\usepackage{colortbl} % cores em tabelas
\newcolumntype{Z}{|>{\columncolor[gray]{0.9}}l|} %cor cinza em células
%\usepackage{array} % já incluso na classe
\newcolumntype{L}[1]{>{\raggedright\let\newline\\\arraybackslash\hspace{0pt}}m{#1}}
\newcolumntype{C}[1]{>{\centering\let\newline\\\arraybackslash\hspace{0pt}}m{#1}}
\newcolumntype{R}[1]{>{\raggedleft\let\newline\\\arraybackslash\hspace{0pt}}m{#1}}
\usepackage{multirow} % para juntar duas linhas em uma só

\usepackage{multicol} % para uso de várias colunas

% cores para os links cruzados
\usepackage{color}
\definecolor{rltred}{rgb}{0.2,0,0}
\definecolor{rltgreen}{rgb}{0,0.2,0}
\definecolor{rltblue}{rgb}{0,0,0.2}

\usepackage[colorlinks=true,
            urlcolor=rltblue,       % \href{...}{...} external (URL)
            filecolor=rltgreen,     % \href{...} local file
            linkcolor=rltred,       % \ref{...} and \pageref{...}
            citecolor=rltgreen,
            pdftitle={Exemplo de Uso da Classe Uflamon},
          pdfauthor={Joaquim Quinteiro Uchôa},
          pdfsubject={Este texto tem por objetivo servir de exemplo da classe Uflamon.},
          pdfkeywords={Comunicação Científica. 2. Pesquisa . 3. Pesquisa Científica. 
 					 4. Redação. 5. Monografia.}%
]{hyperref} % para referência cruzadas
%\usepackage{hyperref}            % para referência cruzadas
\usepackage{subfigure}           % figuras dentro de figuras
\usepackage{caption}            % remodelando o formato dos títulos de 
                                 % tabelas e figuras

% configuração padrão do listings   
\lstset{
   language=Java,
   extendedchars=true,
   tabsize=3,
   basicstyle=\footnotesize\ttfamily,
   stringstyle=\em,
   showstringspaces=false 
}

% para referências de acordo com a ABNT
% precisa instalar o abntex2 antes!!!
% http://abntex.codigolivre.org.br/
% comente se pretende usar outro padrão

%abnt-emphasize=bf coloca o título das bibliografias em negrito
%abnt-thesis-year=both
\usepackage[alf,abnt-etal-cite=3,abnt-etal-list=3,abnt-url-package=url,abnt-emphasize=bf]{abntex2cite}

% evite usar o hyperref com abntex, pode dar caca em urls... no linha anterior, informo
% para incluir urls usando o pacote url e não o hyperref
%
% caso queira o hyperref com abntex, comente a linha anterior e descomente a seguinte
%\usepackage[alf,abnt-etal-cite=3,abnt-etal-list=0,abnt-etal-text=emph]{abntex2cite}
%
% caso vc ainda use a versão anterior da abntex, comente a linha incluindo o abntex2cite
% e descomente a próxima linha 
%\usepackage[alf,abnt-etal-cite=3,abnt-etal-list=0,abnt-etal-text=emph]{abntcite}


% redefinindo formatação de títulos de tabelas e figuras


%==============================================================================
% para os fãs do Word, descomente as linhas abaixo
%\sloppy %mais espaço entre as linhas
%\usepackage{identfirst} %identando-se a primeira linha de cada seção
%\noindentfirst % Tire o comentário para manter o padrão do LaTeX.

%==============================================================================
% definido comandos na monografia - não é necessário na sua monografia 
% apenas para exemplificar a definição de novos comandos
\newcommand{\defs}[1]{\textsl{#1}}


% Especificando hifenizações que por ventura LaTeX não saiba fazer
% Por padrão 99,9% dos termos em português devem ser hifenizados corretamente.
\hyphenation{hardware software Li-nux am-bien-te diag-nos-ti-car coor-de-na-ção 
FAE-PE Recovery TelEduc Williams UFLA}

%==============================================================================
% Dados da monografia, capa: autor, titulo, banca, etc... - SUBSTITUA DE ACORDO
%==============================================================================
\author{Joaquim Quinteiro Uchôa}
\title{Teste teste}
\subtitle{Exemplo para os Usuários}
\engtitle{Use of Uflamon Class}
\engsubtitle{Sample for Users}
\edicao{3$^a$ edição revista, atualizada e ampliada}
\date{2016}
\tipo{Tese apresentada à Universidade Federal de Lavras, como parte das exigências do Programa de Pós-Graduação em Monografia, área de concentração em TCC, para a obtenção do título de Doutor.}
% use \orientador ou \orientadora quando for o caso
\orientador{Prof. DSc. José Orientador}
%\orientadora{}
% use \coorientador ou \coorientadora quando for o caso
\coorientadora{Prof. DSc. Maria Orientadora } % comente se não tiver coorientador
%\coorientador{}
\local{Lavras -- MG}
\bancaum{Prof. MSc. Antônio Banca Um}{UFM}
\bancadois{Prof. DSc. João Banca Dois}{FCO} % comente se sua banca tiver só um professor
\bancatres{Profa. Esp. Eliza Banca Três}{BELMIS}
\bancaquatro{Prof. Esp. Carlos Banca Quatro}{IBGPLUS}
\defesa{30 de Fevereiro de 2016}
%==============================================================================
%##################################################
% Dados para Ficha catalográfica, gerada pelo sistema da Biblioteca da UFLA
% http://www.biblioteca.ufla.br/FichaCatalografica/
% dados para ficha catalográfica
% Elaboração da Ficha Catalográfica
\preparofichacat{Ficha catalográfica elaborada pela Coordenadoria de Processos Técnicos \\ da Biblioteca Universitária da UFLA}
% primeiro autor - como na primeira linha da ficha catalográfica
\fcautor{Uchôa, Joaquim Quinteiro}
% autores, separados por vírgula - na ficha catalográfica, no formato que
% vem após o título e a barra ("/")
\fcautores{Joaquim Quinteiro Uchôa}
% caso trabalho seja ilustrado (figuras, gráficos, tabelas, etc.), 
% então informar por meio do comando a seguir
% caso não seja ilustrado, basta comentá-lo
\fcilustrado{il.}
% dados da edição para a ficha 
\fcedicao{2$^a$ ed. rev., atual. e ampl.}
% tipo do trabalho (tese, dissertação, etc.), de acordo com sistema
% de geração de ficha catalográfica
\fctipo{Tese(doutorado)}
% ano da defesa, só precisa informar se for diferente do ano da publicação
% se forem iguais, comente a linha a seguir
\fcdatadefesa{2016}
% preencher aqui com os dados de catalogação gerados pelo sistema
\fccatalogacao{1. TCC. 2. Monografia. 3. Dissertação. 4. Tese. 5. Trabalho Científico – Normas. I. Universidade Federal de Lavras. II. Título.}
\fcclasi{808.066}

%##################################################

%\antesfichacat{\noindent Para citar este documento: \\UNIVERSIDADE FEDERAL DE LAVRAS. Biblioteca Universitária. \textbf{Manual de normalização e estrutura de trabalhos acadêmicos: TCC, monografias, dissertações e teses}. 2. ed. rev., atual. e ampl. Lavras, 2015. Disponível em: \url{http://www.biblioteca.ufla.br/wordpress/wpcontent/uploads/bdtd/manual_normalizacao_UFLA.pdf}. Acesso em: data de acesso.}

%\depoisfichacat{\noindent A reprodução e a divulgação total ou parcial deste trabalho são autorizadas, por qualquer meio convencional ou eletrônico, para fins de estudo e pesquisa, desde que citada a fonte.\\
%\newline
%{\small Este documento possui páginas em branco para facilitar a impressão frente-e-verso.}}

%##################################################

%##################################################

% para os exemplos do manual
%\newenvironment{exemplomanual}{
%\vspace{0.5cm}
%\noindent\begin{minipage}{\textwidth}
%\noindent\rule{\textwidth}{0.5pt}
%\vspace{-1cm}
%\begin{flushleft}
%}{
%\end{flushleft}
%\vspace{-0.6cm}
%\noindent\rule{\textwidth}{0.5pt}
%\vspace{0.3cm}
%\end{minipage}
%}

%\newenvironment{exemplomanuallista}{
%\vspace{0.3cm}
%\noindent\begin{minipage}{\textwidth - 0.5cm}
%\noindent\rule{\textwidth}{0.5pt}
%\vspace{-1cm}
%\begin{flushleft}
%}{
%\end{flushleft}
%\vspace{-0.6cm}
%\noindent\rule{\textwidth}{0.5pt}
%\vspace{0.3cm}
%\end{minipage}
%}

% por conta de alguns exemplos
%\usepackage{setspace}

%##################################################

% se vc já defendeu e tem o arquivo escaneado da folha de rosto, 
% descomente e altere o nome do arquivo
%\folhaAprovacaoAssinada{folharosto}

% Aqui começa o documento propriamente dito
\begin{document}

\maketitle

\dedic{Dedico aos meus pais, Elisa e Gilberto, ao meu irmão Bruno,e a todos aqueles que me inspiraram durante essa caminhada.}     % Dedicatórias


\thanks{
Essa era a página que eu tanto desejava escrever. Uma  representação de mais uma etapa que está sendo concluída, de um novo ciclo que está prestes a começar. Irei chamar essa de página de página da gratidão. Há muitas pessoas que sou grato por ter chegado até aqui!
Meus pais, Gilberto e Elisa, por abrirem mão de muitas coisas, desde o início, para investir na minha educação, nos meus estudos. Ainda criança era bem difícil de entender o porquê, mas hoje tenho absoluta certeza de que cada palavra de incentivo dos dois foi o que me trouxe até aqui! Meu irmão, Bruno, por estar sempre presente, contagiando ainda mais os meus dias de alegria!
À minha família, obrigado pelo carinho! Aos meus avós, pelas rezas, pelo cuidado e preocupação sobre meu bem estar. O neto está quase voltado!

Durante minha ida à Lavras, muitos amigos ficaram. Obrigado pela paciência e por entender essa distância. Luiza Fujii e Marcell: estão sempre em meus pensamentos! Dina, Carol, Arthur e Gabriel: obrigado por todo o apoio durante esse último ano!

Durante minha trajetória, tive o prazer de fazer muitos amigos. Minhas amigas da República Mesadose: agradeço pelas festas! Aos meus amigos do Palácio da Cartas, obrigado pela companhia!\

Ao escolher Ciência da Computação, meu objetivo durante o curso foi, além de aperfeiçoar em assuntos de meu interesse, foi de, alguma forma , transmitir os conhecimentos que adquiri durante toda a jornada. Sou muito grato por todos os grupos de estudos, pesquisa e extensão que passei, entre eles, CACC, Comp Júnior, PETI, Alcance, DGTI, SETI. Extende-se a todos os amigos que fiz em cada um deles. Agradeço ao Google Developers Group, pela confiança em poder tocar esse projeto em Lavras, tentando tornar a comunidade de desenvolvedores da cidade mais forte e por ajudar a incentivar o empoderamento feminino na tecnologia.
 
Uma boa parte da minha graduação foi marcada pelo intercâmbtio nos EUA, e lá tive o imenso prazer de conhecer pessoas que foram fundamentais e que cada um tem todo o meu respeito. Agradeço aos professores da Auburn University at Montgomery e da Arizona State Univesity. E, principalmente, os amigos que conquistei por lá. Foram aprendizados para a vida toda. Todo dia era dia de dizer "It's gonna be the best day of my life", e foi.

Agradeço aos mestres e doutores que foram meus professores na Universidade Federal de Lavras. Todo o conhecimento que adquiri foram essenciais por eu ser quem eu sou. Obrigado pela dedicação e paciência!

Em especial, agradeço o professor André Pimenta Freire, por me orientar em diversos projetos na universidade. Encontrei durante os projetos o que realmente eu gosto de fazer e estudar, e o que quero dedicar boa parte dos meus estudos daqui em diante: a experiência do usuário.

Agradeço aos meus companheiros de trabalho da Progolden, onde realizei o meu estágio, resultando o trabalho descrito neste documento. Aprendi a rabalhar em grupo vi de perto como um projeto sai do papel e se transforma em um software. E e assim, daqui em diante, que espero que a vida me msorte vários projetos sendo executados e que, principalmente, mude a vida das pessoas.

Assim, dedico à todos(as) os(as) corajosos(as) que colocaram na cabeça a ideia de fazer Ciência da Computação!
}        


	
	
	% Agradecimentos

\epigrafe{ % citação opcional
What you want, what you need.\\
And what'd you come here for?\\
Raise up to your ability\\
You gotta focus on your ability!\\
(Houdini - Foster the People)}

% palavras-chave
\palchaves{Resumo. Palavras. Representativas.}
\resumo{O resumo deve conter palavras representativas do conteúdo do trabalho, localizadas abaixo do resumo, separadas por dois espaços, antecedidas da expressão palavras-chave. Essas palavras representativas são grafadas com a letra inicial em maiúscula, separadas entre si por ponto.}  % Resumo (digite aqui o resumo)

% keywords devem vir antes do abstract
\keywords{Summary. Words. Representative.} % keywords
\abstract{The abstract should contain representative words of the work content, located below the abstract, separated by two spaces, preceded by the keyword expression. These representative words are spelled with the first letter capitalized, separated by point.}

%##################################################

% Dados do guia
%\begin{titlepage}
%\pagestyle{empty}
%\renewcommand{\baselinestretch}{1}
%\enlargethispage{1.5cm}
%\input{reitoria}
%\cleardoublepage
%\end{titlepage}

%##################################################

% descomente para habilitar a lista desejada
\listoffigures                             % Lista de Figuras
%\listofilustracoes
%\listofgraficos							   % Lista de Gráficos
\listoftables                              % Lista de Tabelas
\listofquadros							   % Lista de Quadros
%\listofexemplos
%\listofteoremas
\tableofcontents                           % Sumário

\clearpage

\pagestyle{ufla}

%==============================================================================
% incluindo os capitulos
\chapter{INTRODUÇÃO}

texto

Entretanto, uma das grandes limitações da impressão 3D, principalmente
do tipo \textit{Fused Deposition Modeling}, é o tempo de impressão, que ainda 
limita muitoo tamanho de peças impressas em um tempo razoável, 
geralmente sendo necessário reduzir muito a resolução da impressão.

Existe hoje, dentro da academia e das comunidades "faça você mesmo", uma busca por 
impressoras capazes de imprimir cada vez mais rápido mantendo a qualidade 
de impressão. Além  da possível diminuição do tempo de impressão, 
além disso a capacidade de imprimir velozmente acaba proporcionando 
uma capacidade de aumentar a qualidade de impressão proporcional à diferença
entre a velocidade máxima e a velocidade de impressão.

Portanto, vê-se  relevante à procura por técnicas que permitem capacidades 
superiores de qualidade e velocidade de impressão, que flexibilizam a 
tecnologia e aumentam a capacidade da utilização comercial viável da tecnologia.

\include{elementos}
\chapter{CONCLUSÃO}
Esta monografia apresentou um método de controle de trajetória para sistemas de impressão 3D, focado em aprimorar a precisão do posicionamento da ferramenta. A implementação do método demonstrou uma diminuição notável no desvio do caminho da ponta quando comparado ao caminho simulado sem controle de trajetória.

Os resultados obtidos por meio de simulações confirmaram a capacidade do controle em atenuar as complexidades dinâmicas do sistema. A simulação de referência, juntamente com as simulações de parâmetros variados, validou a eficácia do algoritmo de controle sob diferentes condições operacionais.

A análise dos parâmetros do sistema, incluindo frequência natural, coeficiente de amortecimento, aceleração de entrada, passo de tempo e velocidade desejada, permitiu quantificar sua influência na resposta do controle e teve resultados esperados indicando uma concordância verossímil com os modelos estabelecidos.

A pesquisa identificou a importância de ajustar o passo de tempo de acordo com a frequência natural para uma representação mais fiel da resposta do sistema e indicou a execução do controle de trajetória em etapas de diferentes passos de tempo como uma abordagem efetiva para aprimorar o processo de controle.

Em síntese, as contribuições deste trabalho para o campo do controle de trajetória são evidenciadas pela melhoria na precisão da impressão 3D dada pelas simulações e contribui para a exploração de métodos iterativos e de programação linear, estabelecendo um ponto de partida para futuros avanços na otimização de sistemas de controle no contexto de impressoras 3D.

%==============================================================================
% Incluindo bibliografia
%\bibliographystyle{plain}             % estilo para labels em numeros
%\bibliographystyle{alpha}             % estilo para labels em iniciais
\bibliographystyle{abntex2-alf}           % estilo para referências usando ABNT, 
                                       % precisa instalar o abntex para usar!!!

%inclui Referências Bibliográficas
%inclui Referências Bibliográficas
\referencias
\bibliography{refbib}			% arquivo exemplo refbib.bib
%==============================================================================
% Incluindo anexos num1erados com letras maiusculas.
%\apendices
\apendice{Tarefas da Avaliação Heurística}
\label{cap:apendice}



%==============================================================================
% Fim do texto
\end{document}
