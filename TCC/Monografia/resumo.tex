\palchaves{Manufatura aditiva Modelagem por Fusão e Deposição (FDM) Geração de comandos Impressão 3D Algoritmos de controle Modelagem dinâmica Matlab fmincon}
\resumo{%
Este trabalho aborda a geração de comandos em impressoras 3D
utilizando o método de manufatura aditiva conhecido como
\textit{Fused Deposition Modeling} (FDM).
% 
O objetivo principal é investigar e desenvolver metodologias para
atuação de controle na geração de comandos em impressoras 3D de
forma a possibilitar maiores velocidades e garantindo a precisão
dimensional das peças produzidas.
% 
O referencial teórico apresenta os fundamentos da manufatura aditiva,
o FDM e técnicas de controle, trajetória e modelagem dinâmica do
sistema.
% 
Os resultados mostram correlações entre as variáveis de entrada,
influências do modelo dinâmico e a performance computacional.
%
Possíveis abordagens, como divisão de etapas de solução,
são sugeridas.
% 
Conclui-se que a geração de comandos é crucial para a qualidade e
eficiência da manufatura aditiva por FDM, e pesquisas futuras
podem explorar soluções mais robustas.
% 
Este estudo contribui para o aprimoramento da manufatura aditiva,
incentivando pesquisas adicionais e impulsionando sua
aplicação em diversos setores industriais.
}  

\keywords{Additive manufacturing Fused Deposition Modeling (FDM) Command generation 3D printing Control algorithms Dynamic modeling Matlab fmincon} 
\abstract{This work addresses the command generation in 3D printers
using the additive manufacturing method known as Fused Deposition Modeling (FDM). The main objective is to investigate and develop control methodologies for command generation in 3D printers, enabling higher speeds while ensuring dimensional accuracy of the produced parts. The theoretical framework presents the fundamentals of additive manufacturing, FDM, and control techniques, trajectory analysis, and dynamic modeling of the system. The methodology involves the use of Matlab software and the fmincon function for command generation. The results demonstrate correlations between input variables, influences of the dynamic model, and computational performance. Difficulties such as the objective function, nonlinear constraints of the base, and main variables are discussed. Possible approaches, such as stage division for problem solving, are suggested. It is concluded that command generation is crucial for the quality and efficiency of FDM additive manufacturing, and future research can explore more robust solutions. This study contributes to the improvement of additive manufacturing, encouraging further research and promoting its application in various industrial sectors.
}
