\palchaves{Manufatura aditiva Modelagem por Fused Deposition Modeling FDM Geração de Trajetória Impressão 3D Algoritmos de controle Modelagem dinâmica}
\resumo{%
A manufatura aditiva, com ênfase na impressão 3D e no método "Fused Deposition Modeling" (FDM), destaca-se como uma tecnologia altamente promissora para a produção de peças complexas em quantidades reduzidas. Ela impulsiona a iterabilidade e a produção descentralizada sob demanda, encontrando aplicação em diversos setores, como engenharia, medicina e a indústria aeroespacial.
%
O objetivo principal do trabalho é investigar e desenvolver uma metodologia para
atuação de controle na geração de comandos em impressoras 3D de
forma a possibilitar maiores velocidades e garantindo a precisão
dimensional das peças produzidas.
%
O desenvolvimento desse controle incorpora um algoritmo iterativo que minimiza os desvios entre o trajeto desejado para a impressora e o trajeto efetivamente percorrido, levando em consideração a dinâmica da impressora. Isso resulta em peças impressas de maior qualidade, permitindo aos usuários selecionar velocidades de impressão mais elevadas sem comprometer a qualidade do produto final, em comparação com a ausência desse método.
%
O algoritmo foi submetido a testes de sensibilidade de parâmetros para avaliar seu desempenho sob diferentes condições. Os resultados desses testes foram satisfatórios e indicaram a capacidade do método de minimizar os desvios da impressora, com base em uma modelagem precisa. Além disso, o método apresenta potencial para expansão, possibilitando a consideração de nuances do modelo em várias regiões da impressora, aprimorando ainda mais sua eficácia.
}  

\keywords{Additive manufacturing Fused Deposition Modeling (FDM) Trajectory generation 3D printing Control algorithms Dynamic modeling} 
\abstract{%
Additive manufacturing, with an emphasis on 3D printing and the "Fused Deposition Modeling" (FDM) method, stands out as a highly promising technology for the production of complex parts in reduced quantities. It drives iterability and decentralized on-demand production, finding applications in various sectors, such as engineering, medicine, and the aerospace industry.
%
The primary goal of this work is to investigate and develop a methodology for control action in generating commands in 3D printers to enable higher speeds while ensuring the dimensional accuracy of the produced parts.
%
The development of this control incorporates an iterative algorithm that minimizes deviations between the desired path for the printer and the path actually traveled, taking into account the printer's dynamics. This results in higher quality printed parts, allowing users to select higher printing speeds without compromising the quality of the final product compared to the absence of this method.
%
The algorithm underwent parameter sensitivity tests to assess its performance under different conditions. The results of these tests were satisfactory and indicated the method's capability to minimize printer deviations based on accurate modeling. Furthermore, the method has potential for expansion, allowing for the consideration of nuances in the model in various printer regions, further enhancing its effectiveness.
}
