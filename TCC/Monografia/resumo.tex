\palchaves{Manufatura aditiva Modelagem por Fusão e Deposição (FDM) Geração de comandos Impressão 3D Algoritmos de controle Modelagem dinâmica Matlab fmincon}
\resumo{%
A manufatura aditiva, com ênfase na impressão 3D e no método "Fused Deposition Modeling" (FDM), destaca-se como uma tecnologia altamente promissora para a produção de peças complexas em quantidades reduzidas. Ela impulsiona a iterabilidade e a produção descentralizada sob demanda, encontrando aplicação em diversos setores, como engenharia, medicina e a indústria aeroespacial.
%
O objetivo principal do trabalho é investigar e desenvolver uma metodologia para
atuação de controle na geração de comandos em impressoras 3D de
forma a possibilitar maiores velocidades e garantindo a precisão
dimensional das peças produzidas.
%
O desenvolvimento desse controle incorpora um algoritmo iterativo que minimiza os desvios entre o trajeto desejado para a impressora e o trajeto efetivamente percorrido, levando em consideração a dinâmica da impressora. Isso resulta em peças impressas de maior qualidade, permitindo aos usuários selecionar velocidades de impressão mais elevadas sem comprometer a qualidade do produto final, em comparação com a ausência desse método.
%
O algoritmo foi submetido a testes de sensibilidade de parâmetros para avaliar seu desempenho sob diferentes condições. Os resultados desses testes foram satisfatórios e indicaram a capacidade do método de minimizar os desvios da impressora, com base em uma modelagem precisa. Além disso, o método apresenta potencial para expansão, possibilitando a consideração de nuances do modelo em várias regiões da impressora, aprimorando ainda mais sua eficácia.
}  

\keywords{Additive manufacturing Fused Deposition Modeling (FDM) Command generation 3D printing Control algorithms Dynamic modeling Matlab fmincon} 
\abstract{This work addresses the command generation in 3D printers
using the additive manufacturing method known as Fused Deposition Modeling (FDM). The main objective is to investigate and develop control methodologies for command generation in 3D printers, enabling higher speeds while ensuring dimensional accuracy of the produced parts. The theoretical framework presents the fundamentals of additive manufacturing, FDM, and control techniques, trajectory analysis, and dynamic modeling of the system. The methodology involves the use of Matlab software and the fmincon function for command generation. The results demonstrate correlations between input variables, influences of the dynamic model, and computational performance. Difficulties such as the objective function, nonlinear constraints of the base, and main variables are discussed. Possible approaches, such as stage division for problem solving, are suggested. It is concluded that command generation is crucial for the quality and efficiency of FDM additive manufacturing, and future research can explore more robust solutions. This study contributes to the improvement of additive manufacturing, encouraging further research and promoting its application in various industrial sectors.
}
