\palchaves{Impressão 3D Programação Não-linear Manufatura aditiva Fused Deposition Modeling FDM Algoritimo Iterativo Otimização de Trajetória}
\resumo{%
A manufatura aditiva, com ênfase na impressão 3D e no método  de Modelagem de Deposição Fundida do inglês \textit{"Fused Deposition Modeling"} (FDM), destaca-se como uma tecnologia altamente promissora para a produção de peças complexas em quantidades reduzidas. Ela impulsiona a iterabilidade e a produção descentralizada, encontrando aplicação em diversos setores, como engenharia, medicina e a indústria aeroespacial.
%
O objetivo principal do trabalho é investigar e desenvolver uma metodologia para otimizar a trajetória em impressoras 3D de forma a reduzir as vibrações, permitindo maiores velocidades e uma maior precisão dimensional das peças produzidas.
%
O desenvolvimento desse controle incorpora um algoritmo iterativo que minimiza os desvios entre o trajeto desejado para a impressora e o trajeto efetivamente percorrido, levando em consideração a dinâmica da impressora. Isso resulta em peças impressas de maior qualidade, permitindo aos usuários selecionar velocidades de impressão mais elevadas sem comprometer a qualidade do produto final, em comparação com a ausência desse método.
%
O algoritmo foi submetido a diferentes testes, variando seus parâmetros de entrada, para avaliar seu desempenho sob diferentes condições. Os resultados desses testes foram satisfatórios e indicaram a capacidade do método de minimizar os desvios da impressora, com base em uma modelagem precisa. Além disso, o método apresenta potencial para expansão, possibilitando a consideração de nuances do modelo em várias regiões da impressora, aprimorando ainda mais sua eficácia.
}  

\keywords{3D Printing Non-linear Programming Additive Manufacturing Fused Deposition Modeling FDM Iterative Algorithm Trajectory Optimization} 
\abstract{%
Additive manufacturing, with a focus on 3D printing and the Fused Deposition Modeling (FDM) method, stands out as a highly promising technology for the production of complex parts in small quantities. It promotes iterability and decentralized production, finding application in various sectors such as engineering, medicine, and the aerospace industry.
%
The main goal of this work is to investigate and develop a methodology to optimize the trajectory in 3D printers in order to reduce vibrations, allowing for higher speeds and greater dimensional accuracy of the produced parts.
%
The development of this control involves an iterative algorithm that minimizes the deviations between the desired path for the printer and the actual path taken, taking into account the dynamics of the printer. This results in higher quality printed parts, allowing users to choose higher printing speeds without compromising the quality of the final product, compared to the absence of this method.
%
The algorithm was subjected to different tests, varying its input parameters, to evaluate its performance under various conditions. The results of these tests were satisfactory and demonstrated the method's ability to minimize printer deviations, based on accurate modeling. Furthermore, the method shows potential for expansion, allowing for the consideration of model nuances in various regions of the printer, further enhancing its effectiveness.
}
