\palchaves{Palavras chave}
\resumo{
Com a crise global logística causada pela pandemia, uma das tecnologias que mais se destacou foi a 
participação da manufatura aditiva para uma produção flexível e descentralizada. A impressão 3D tem
 ganhado cada vez mais espaço em aplicações variadas, como no setor aeroespacial, automobilístico e
  principalmente aplicações em prototipagem rápida, cada vez mais acessível.
O presente trabalho estabelece um modelo dinâmico base para uma impressora 3D e utiliza uma função
 disponível no Matlab, FMINCON, para realizar uma otimização dos comandos dinâmicos de forma a reduzir 
 o desvio da trajetória desejada, de forma iterativa.
Os resultados obtidos mostram o potencial da técnica, entretanto necessidade de uma execução melhor 
para poder ser testado de forma mais prática e atrelado a outras técnicas como a de pré processamento do 
arquivo Gcode.

}  % Resumo (digite aqui o resumo)


\keywords{Keywords} % keywords
\abstract{
With the global logistical crisis caused by the pandemic, one of the
 technologies that stood out the most was the participation of additive 
 manufacturing for flexible and decentralized production. 3D printing has 
 gained more and more space in various applications, such as in the aerospace, 
 automotive and especially rapid prototyping applications, which are increasingly accessible.
The present work establishes a base dynamic model for a 3D printer and uses a
 function available in Matlab, FMINCON, to perform an optimization of dynamic 
 commands in order to iteratively reduce the deviation from the desired trajectory.
The results obtained show the potential of the technique, however, the need for a
 better execution to be able to be tested in a more practical way and linked to other techniques 
 such as the pre-processing of the Gcode file.
}
