\chapter{INTRODUÇÃO}

Introdução



Entretanto, uma das grandes limitações da impressão 3D, principalmente
do tipo \textit{Fused Deposition Modeling}, é o tempo de impressão, que ainda 
limita muitoo tamanho de peças impressas em um tempo razoável, 
geralmente sendo necessário reduzir muito a resolução da impressão.

Existe hoje, dentro da academia e das comunidades "faça você mesmo", uma busca por 
impressoras capazes de imprimir cada vez mais rápido mantendo a qualidade 
de impressão. Além  da possível diminuição do tempo de impressão, 
além disso a capacidade de imprimir velozmente acaba proporcionando 
uma capacidade de aumentar a qualidade de impressão proporcional à diferença
entre a velocidade máxima e a velocidade de impressão.

Portanto, vê-se  relevante à procura por técnicas que permitem capacidades 
superiores de qualidade e velocidade de impressão, que flexibilizam a 
tecnologia e aumentam a capacidade da utilização comercial viável da tecnologia.


A manufatura aditiva tem se mostrado uma tecnologia promissora para a fabricação de peças e componentes em diversas áreas, como engenharia, medicina e indústria aeroespacial. 
Entre os diferentes métodos de manufatura aditiva, destaca-se o Fused Deposition Modeling (FDM), 
que utiliza a deposição de material fundido para construir objetos camada por camada.

No entanto, a qualidade e a eficiência do processo de manufatura aditiva ainda são desafios a serem superados. 
Nesse sentido, é necessário um controle preciso da geração de comandos para a impressora 3D, visando otimizar 
a velocidade de movimentação e garantir a precisão dimensional das peças produzidas.

O presente trabalho tem como objetivo investigar e desenvolver metodologias para a geração de comandos em 
impressoras 3D utilizando o método FDM. Para isso, serão abordados conceitos teóricos fundamentais, como a 
utilização de algoritmos de controle, a análise de trajetórias e a modelagem dinâmica do sistema.

No capítulo 2, será apresentado o referencial teórico, que abrange desde os princípios básicos da 
manufatura aditiva até os conceitos específicos do Fused Deposition Modeling. Serão discutidos também os 
métodos de geração de comando, como o look ahead, as curvas de velocidade trapezoidal e a integração implícita
utilizando programação não linear e colocação.

No capítulo 3, será descrita a metodologia utilizada para o desenvolvimento do trabalho. Serão apresentadas as 
ferramentas utilizadas, como o software Matlab e a função fmincon, além das etapas para a geração de comandos 
e a modelagem dinâmica da impressora 3D.

No capítulo 4, serão apresentados os resultados obtidos e discutidos. Serão analisadas as correlações entre as 
variáveis de entrada, as influências do modelo dinâmico e a performance computacional do sistema. 
Serão abordadas também as dificuldades encontradas durante o desenvolvimento do trabalho e possíveis abordagens para superar essas dificuldades, como a combinação com outros algoritmos.

Por fim, no capítulo 5, serão apresentadas as conclusões do trabalho, destacando as principais contribuições e 
apontando possíveis direções para pesquisas futuras. O objetivo final é fornecer subsídios para o aprimoramento
da geração de comandos em impressoras 3D, visando melhorar a qualidade e eficiência do processo de manufatura aditiva.

