\chapter{INTRODUÇÃO}
A manufatura aditiva tem se mostrado uma tecnologia promissora para a fabricação de peças e componentes em
diversas áreas, como engenharia, medicina e indústria aeroespacial. Com a crise global logística causada
pela pandemia, a manufatura aditiva ganhou destaque devido à sua capacidade de produção flexível e 
descentralizada. Nesse contexto, a impressão 3D, em particular o método 
\textit{Fused Deposition Modeling} (FDM), tem se tornado cada vez mais relevante, encontrando aplicações 
variadas em setores como aeroespacial, automobilístico e prototipagem rápida, tornando-se mais acessível.

A modelagem digital, por exemplo \textit{Computer Aided Design} (CAD), anda lado a lado ao processo de impressão 3D
e desempenha um papel fundamental no processo de criação e distribuição, permitindo a concepção de modelos 
tridimensionais precisos que podem ser compartilhados e reproduzidos de maneira descentralizada. 
Para se imprimir um desses modelos é feita uma preparação por meio de um \textit{software} de fatiamento 
(\textit{slicer}), que divide o modelo em camadas e gera os comandos necessários para a impressora 3D. 
A partir destes comandos a impressora interpreta os passos que deverão ser tomados e quando cada um deles 
deve ser realizado. Entre a interpretação e execução destes comandos existe uma diversidade de processos 
intermediarios que tem impacto direto na qualidade e na velocidade de impressão.

No entanto, uma das limitações significativas da impressão 3D, especialmente do tipo FDM, é o tempo de 
impressão, que ainda restringe o tamanho das peças produzidas em um período razoável. Frequentemente, é 
necessário utilizar camadas e linhas mais grossas para compensar esse aspecto, diminuindo a habilidade de
se reproduzir detalhes menores. Diante disso, existe uma procura por maneiras de se imprimir mais rapidamente, 
sem comprometer a qualidade.
Assim, é relevante explorar técnicas que permitam alcançar capacidades superiores de qualidade e 
velocidade de impressão, flexibilizando a tecnologia e ampliando sua aplicação comercial viável. 

Este trabalho tem como objetivo investigar e desenvolver metodologias para atuação de controle na geração de
comandos em impressoras 3D utilizando o método FDM. Será abordado o referencial teórico, contemplando os 
princípios básicos da manufatura aditiva, os conceitos específicos do \textit{Fused Deposition Modeling}, 
os métodos de geração de comando, como \textit{look ahead}, curvas de velocidade trapezoidal e o estado da 
arte quanto ao desenvolvimento de metodos de controle que buscam aprimorar a qualidade e diminuir o tempo 
necessário para imprimir uma peça, como \textit{InputShaping}.

Será descrita a metodologia utilizada, descrevendo as
etapas para a geração de comandos e a modelagem dinâmica da impressora 3D. Os resultados obtidos serão 
apresentados e discutidos, analisando as correlações entre as variáveis de entrada, as influências do 
modelo dinâmico e a performance computacional. Serão abordadas as dificuldades encontradas durante o 
desenvolvimento do trabalho e possíveis abordagens para superá-las, como a combinação com outros algoritmos.

Por fim, serão apresentadas as conclusões, destacando as contribuições principais e apontando possíveis 
direções para pesquisas futuras. A proposta é fornecer subsídios para aprimorar a geração de comandos em 
impressoras 3D, visando melhorar a qualidade e eficiência do processo de manufatura aditiva.

\section{Impressora 3d}

Extrusora: A extrusora é responsável por derreter o filamento de material termoplástico e extrudá-lo 
em forma de filamento derretido. 
Ela consiste em um bico aquecido (hotend) que funde o material e um motor que empurra o 
filamento através do bico. Alguns modelos mais avançados podem ter extrusoras duplas para 
imprimir com materiais diferentes ou suportes solúveis.

Mesa de impressão: A mesa de impressão é a superfície onde o objeto está sendo construído. 
Ela é aquecida em muitas impressoras FDM para ajudar a aderência do material à superfície. 
Além disso, algumas mesas de impressão têm características especiais, como superfícies 
texturizadas ou magnéticas, para facilitar a aderência e a remoção do objeto após a conclusão.

Plataforma de construção: A plataforma de construção é o suporte físico onde a mesa de impressão é montada. 
Ela pode ser ajustada em altura para nivelar a superfície de impressão e garantir que a primeira camada do 
objeto seja depositada com precisão.

Motor de movimento: Impressoras 3D FDM possuem motores de movimento que controlam a posição da extrusora e 
da mesa de impressão ao longo dos eixos X, Y e Z. Geralmente são motores de passo e seus movimentos de rotação
são geralmente convertidos em movimentos lineares através de correias ou parafusos de rosca trapezoidal.

Filamento: O filamento é o material de alimentação para a impressora 3D. Ele é um longo fio de plástico 
termoplástico que é inserido na extrusora e derretido durante o processo de impressão. Os filamentos vêm 
em várias cores e tipos de material, dependendo do objeto a ser impresso.

Software de fatiamento: Para imprimir um objeto em uma impressora 3D FDM, você precisa de um software de fatiamento. 
Este software converte modelos 3D em camadas finas onde é definido um percurso preenchendo essas camandas junto
a outros comandos e configurações, como a temperatura do bico injetor, as velocidades máximas desejadas de cada movimento,
a unidade utilizada, entre outras configurações e comandos. Estes comandos descritos através de Gcode e enviado para a impressora. 

Sistema de controle e geração de comando: A eletrônica de controle inclui a placa-mãe da impressora, que recebe comandos do 
software, em geral no formato Gcode, e os traduz em movimentos dos motores, controle de temperatura da extrusora e da mesa de impressão, 
velocidade dos ventiladores entre outros acessórios. 
Ela também pode ter uma tela de exibição e controles para operação manual.