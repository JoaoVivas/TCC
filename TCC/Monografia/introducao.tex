\chapter{INTRODUÇÃO}
Com a crise global logística causada pela pandemia, uma das tecnologias
que mais se destacou foi a participação da manufatura aditiva para uma
produção flexível e descentralizada. A impressão 3D tem ganhado cada vez
mais espaço em aplicações variadas, como no setor aeroespacial,
automobilístico e principalmente aplicações em prototipagem rápida,
cada vez mais acessível.

Entretanto, uma das grandes limitações da impressão 3D, principalmente
do tipo Fused Deposition Modeling, é o tempo de impressão, que ainda 
limita muitoo tamanho de peças impressas em um tempo razoável, 
geralmente sendo necessário reduzir muito a resolução da impressão.

Existe hoje, dentro da academia e das comunidades maker, uma busca por 
impressoras capazes de imprimir cada vez mais rápido mantendo a qualidade 
de impressão. Além  da possível diminuição do tempo de impressão, 
além disso a capacidade de imprimir velozmente acaba proporcionando 
uma capacidade de aumentar a qualidade de impressão proporcional à diferença
entre a velocidade máxima e a velocidade de impressão.

Portanto, vê-se  relevante à procura por técnicas que permitem capacidades 
superiores de qualidade e velocidade de impressão, que flexibilizam a 
tecnologia e aumentam a capacidade da utilização comercial viável da tecnologia.
