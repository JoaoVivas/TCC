%==============================================================================
% Dados da monografia, capa: autor, titulo, banca, etc... - SUBSTITUA DE ACORDO
%==============================================================================


\author{Joao Vivas Cisalpino}
\title{CONTROLE DE VELOCIDADE EM  IMPRESSORAS 3D UTILIZANDO O PRINCÍPIO DE 
FEEDFORWARD PARA OTIMIZAR A VELOCIDADE E QUALIDADE DE IMPRESSÃO.
}
%\subtitle{Exemplo para os Usuários}
%\engtitle{Use of Uflamon Class}
% \engsubtitle{Subtitulo teste}
%\edicao{3$^a$ edição revista, atualizada e ampliada}
\date{2023}
\tipo{Monografia apresentada à Universidade Federal de Lavras,
como parte das exigências Curso de Engenharia Mecânica,
para a obtenção do título de Bacharel.}
% use \orientador ou \orientadora quando for o caso
\orientador{Prof. Dr. Wander Gustavo Rocha Vieira}
%\orientadora{}
% use \coorientador ou \coorientadora quando for o caso
%\coorientadora{Camila Rodrigues Dias} % comente se não tiver coorientador
%\coorientador{}
\local{Lavras - MG}
\bancaum{Ainda não definido}{UFLA}
\bancadois{Ainda não definido}{UFLA?} % comente se sua banca tiver só um professor
\bancatres{Ainda não definido 3}{BELMIS}
%\bancaquatro{Prof. Esp. Carlos Banca Quatro}{IBGPLUS}
\defesa{Data da defesa 2023}
%==============================================================================
%##################################################
% Dados para Ficha catalográfica, gerada pelo sistema da Biblioteca da UFLA
% http://www.biblioteca.ufla.br/FichaCatalografica/
% dados para ficha catalográfica
% Elaboração da Ficha Catalográfica
\preparofichacat{Ficha catalográfica elaborada pela Coordenadoria de Processos Técnicos \\ da Biblioteca Universitária da UFLA}
% primeiro autor - como na primeira linha da ficha catalográfica
\fcautor{Cisalpino, Joao Vivas Cisalpino}
% autores, separados por vírgula - na ficha catalográfica, no formato que
% vem após o título e a barra ("/")
\fcautores{Joao Vivas Cisalpino}
% caso trabalho seja ilustrado (figuras, gráficos, tabelas, etc.), 
% então informar por meio do comando a seguir
% caso não seja ilustrado, basta comentá-lo
\fcilustrado{il.}
% dados da edição para a ficha 
\fcedicao{1$^a$ ed. rev., atual. e ampl.}
% tipo do trabalho (tese, dissertação, etc.), de acordo com sistema
% de geração de ficha catalográfica
\fctipo{Trabalho de conclusão de curso(Graduação)}
% ano da defesa, só precisa informar se for diferente do ano da publicação
% se forem iguais, comente a linha a seguir
\fcdatadefesa{2023}
% preencher aqui com os dados de catalogação gerados pelo sistema
\fccatalogacao{1. TCC}
\fcclasi{808.066}


%\antesfichacat{\noindent Para citar este documento: \\UNIVERSIDADE FEDERAL DE LAVRAS. Biblioteca Universitária. \textbf{Manual de normalização e estrutura de trabalhos acadêmicos: TCC, monografias, dissertações e teses}. 2. ed. rev., atual. e ampl. Lavras, 2015. Disponível em: \url{http://www.biblioteca.ufla.br/wordpress/wpcontent/uploads/bdtd/manual_normalizacao_UFLA.pdf}. Acesso em: data de acesso.}

%\depoisfichacat{\noindent A reprodução e a divulgação total ou parcial deste trabalho são autorizadas, por qualquer meio convencional ou eletrônico, para fins de estudo e pesquisa, desde que citada a fonte.\\
%\newline
%{\small Este documento possui páginas em branco para facilitar a impressão frente-e-verso.}}