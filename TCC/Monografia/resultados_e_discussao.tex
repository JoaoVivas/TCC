\chapter{resultados e discussão}

\section{Resultados comparando os limites de aceleração}
Utilizando valores suficientemente altos para os limites de 
aceleração da base, o algoritmo é capaz manter a trajetória 
de forma excelente com um desvio inferior a menos de 0,001\%, 
como apresentado nas figuras abaixo:

Entretanto, quando é utilizado valores mais baixos de aceleração 
máxima para as mesmas condições, o algoritmo não é capaz de 
encontrar as soluções de alguns pontos e o resultado acaba 
sendo não utilizável, como apresentado na figura abaixo:

\section{Discussão}
A provável causa para os artefatos com baixos limites de 
aceleração é o tempo objetivo estimado, que pode acabar não 
sendo atingido com os limites impostos. Para remediar este 
problema é necessário um método mais eficaz em estimar um 
tempo total de impressão que possa ser atingido com as 
limitações impostas.

% O wireframe, como representa a Figura \ref{fig:wireframe} é uma representação de baixa fidelidade de uma interface. Foi reproduzido digitalmente.

% \begin{figure}[H]
% \centering
% \includegraphics[width=1\textwidth]{images/wireframe.png}
% \caption{Wireframe inicial}
% \label{fig:wireframe}
% \end{figure}



% \subsubsection{MarvelApp}
%  O MarvelApp\footnote{https://marvelapp.com} permite a criação de protótipos de alta fidelidade de maneira rápida, eficiente e portátil para testes realizados em smartphones. A partir de imagens e mockups, é possível transformá-los em protótipos para qualquer tipo de dispositivo, sem necessidade de codificação.

% \subsubsection{Axure}
% Soluções podem ser prototipadas e validadas pelas pessoas que melhor compreendem seus negócios, produtos e clientes. O Axure\footnote{https://www.axure.com} permite a criação de flowcharts, wireframes, mockups, personas e quadro de ideias. Além disso, permite uma maior interação juntamente com a entrada de dados pelo usuário.



% \begin{table}[H]
% \centering
% \caption{Severidades}
% \label{tab:severidade}
% \begin{tabular}{c|c|c}
%       \hline
%        \rowcolor[gray]{.9}
%       \bf Severidade  & \bf Frequência & \bf Porcentagem  \\
%       \hline
%       \hline
% 0 (Não acredito que seja um problema) & 0          & 0\%         \\
% 1 (Cosmético)                        & 12         & 14,16\%     \\
% 2 (Simples)                          & 71         & 83,78\%     \\
% 3 (Sério)                           & 35         & 41,3\%      \\
% 4 (Catastrófico)                     & 0          & 0\%         \\

% \hline
% \hline
% \bf TOTAL   & \bf 118        & \bf 100\%   \\
%  \hline 
% \end{tabular}
% \end{table}


% \begin{table}[H]
% \centering
% \caption{ Violações categorizadas por Heurísticas}
% \label{tab:heuristicasAfetadas}
% \begin{tabular}{c|c|c}
%       \hline
%       \rowcolor[gray]{.9}
%  \bf Heurística   &  \bf Frequência & \bf Porcentagem  \\
%       \hline
%       \hline
% H1. Visibilidade do status do sistema
% & 10  & 7\%\\

% H2. Correspondência entre o sistema e o mundo real 
% & 15 & 10\% \\  

% H3. Controle do usuário e liberdade                     
% & 9  & 6\%     \\

% H4. Consistência e padrões                          
% & 29  & 19\%      \\

% H5. Prevenção de erros                        
% &15  & 10\%      \\

% H6. Reconhecimento em vez de recordação                 
% &3  &2\%      \\

% H7. Flexibilidade e eficiência de utilização           
% &5  &3\%      \\

% H8. Estética e design minimalista                       
% & 21  & 14\%      \\

% H9. Ajude os usuários a reconhecer, diagnosticar e resolver erros                
% & 15  &10\%      \\

% H10. Ajuda e documentação                       
% &15  &10\%      \\


% \hline
% \hline
% \bf TOTAL   & \bf 137        & \bf 100\%   \\
%  \hline 
% \end{tabular}
% \end{table}












