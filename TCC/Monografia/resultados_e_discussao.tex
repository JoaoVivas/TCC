\chapter{RESULTADOS E DISCUSSÃO}

\section{Resultados comparando os limites de aceleração}
Utilizando valores suficientemente altos para os limites de 
aceleração da base, o algoritmo é capaz manter a trajetória 
de forma excelente com um desvio inferior a menos de 0,001\%, 
como apresentado nas figuras abaixo:

Entretanto, quando é utilizado valores mais baixos de aceleração 
máxima para as mesmas condições, o algoritmo não é capaz de 
encontrar as soluções de alguns pontos e o resultado acaba 
sendo não utilizável, como apresentado na figura abaixo:

\section{Discussão}
A provável causa para os artefatos com baixos limites de 
aceleração é o tempo objetivo estimado, que pode acabar não 
sendo atingido com os limites impostos. Para remediar este 
problema é necessário um método mais eficaz em estimar um 
tempo total de impressão que possa ser atingido com as 
limitações impostas.





