\chapter{RESULTADOS E DISCUSSÃO}

(PUXAR AS PARTES MAIS CONCRETAS DA METODOLOGIA -> A PARTIR DA TABELA DOS VALORES DOS PARÂMETROS)




Este capítulo detalha os resultados alcançados por meio da análise de sensibilidade

% ACHAR OUTRO NOME PARA ANALISE DE SENSIBILIDADE (menos detalhado)

aplicada às simulações no estudo do controle de trajetória em impressoras 3D. Essa análise é essencial para avaliar a 


% eficiência 
% TROCAR PARA ROBUSTEZ POSSIVELMENTE


do método de controle proposto, considerando sua resposta diante de diferentes 

% VARIAÇÕES DOS parâmetros operacionais, COMO POR EXEMPLO ....


Prosseguindo, a análise dos resultados inicia-se com a avaliação da simulação de referência (ENCONTRAR UMA JUSTIFICATIVA PARA A SIMULAÇÃO DE REFERÊNCIA), estabelecendo um ponto de comparação para as demais simulações. Em seguida, detalhamos (DETALHA-SE -> ESCRITA IMPESSOAL) o impacto e as implicações de cada um dos cinco parâmetros escolhidos (FALAR OS PARÂMETROS) na análise de sensibilidade.  

% A investigação destes parâmetros oferece uma visão abrangente sobre como cada elemento influencia o desempenho geral do sistema, permitindo uma compreensão mais profunda e um refinamento do método proposto. (REPENSAR ESSA AQUI -> PRECISA SER CLARO POR QUE A COMPREENSÃO SERIA MAIS PROFUNDA FAZENDO ISSO)

\section{Resultados da Simulação Referência}
Esta seção aborda os resultados obtidos na simulação de referência, e estabelece a referência para a análise de sensibilidade. 


% As análises são (DIVIDAS EM DIFERENTES GRÁFICOS) quebradas em diferentes focos dado os diferentes gráficos dos resultados. Os seguintes focos de análise são estabelecidos:

% (CONVERTER TUDO EM TEXTO CORRIDO)
% \begin{itemize}
%     \item Análise de Velocidades (Figura \ref{fig:t_padr_vels}): Este gráfico exibe as variações das velocidades ao longo do tempo nos eixos X e Y. Ele demonstra a resposta dinâmica tanto da ponta do manipulador quanto da base, conforme a metodologia proposta.

%     \begin{figure}[H]
%         \begin{center}
%         \caption{Caso referência - Comportamento no tempo das velocidades em x e y da ponta e da base}
%         \includegraphics[scale=0.44]{Teste Padrao vels}
%         \label{fig:t_padr_vels}
%         \end{center}
%     \end{figure}
    
%     Com os parâmetros de referência a trajetória obtida é trapezoidal e é definida pelos parâmetros de velocidade desejada, aceleração e pelas distâncias dos movimentos, como comentado no trabalho.
    
%     As linhas amarela e roxa apresentam as curvas de velocidade da base dos eixos x e y respectivamente, estas foram obtidas através da aplicação do método proposto como solução das restrições para a trajetória desejada da ponta, representada pelas linhas azul e laranja.

%     \item Deslocamentos Temporais (Figura \ref{fig:t_padr_des}): Aqui, os deslocamentos nos eixos X e Y são plotados contra o tempo. Esses resultados refletem diretamente a precisão e eficiência do sistema simulado, conforme os critérios estabelecidos na metodologia.

%     \begin{figure}[H]
%         \begin{center}
%         \caption{Caso referência - Comportamento no tempo dos deslocamentos em x e y da ponta e da base}
%         \includegraphics[scale=0.44]{Teste Padrao des}
%         \label{fig:t_padr_des}
%         \end{center}
%     \end{figure}    

%     \item Trajetória Espacial (Figura \ref{fig:t_padr_pos}): Este gráfico ilustra o caminho percorrido nos planos X e Y, comparando a trajetória da ponta com a trajetória de referência. A aderência ao caminho planejado, conforme descrito na metodologia, é fundamental para avaliar a acurácia do sistema.

%     \begin{figure}[H]
%         \begin{center}
%         \caption{Caso referência - Caminho percorrido x vs y da ponta e da base}
%         \includegraphics[scale=0.44]{Teste Padrao pos}
%         \label{fig:t_padr_pos}
%         \end{center}
%     \end{figure}

    
%     Além disso, o tempo total de simulação para esta análise foi de aproximadamente 89,8 segundos. Os vetores de posição empregados na simulação contavam com 241 elementos.
%     Este gráfico é fundamental para entender a eficácia da simulação em atender às restrições estabelecidas e a capacidade do sistema de convergir para uma solução viável em um tempo eficiente.
% \end{itemize}

(MUDAR A ORDEM DAS EXPLICAÇÕES -> DESLOCAMENTO PRIMEIRO DPS VELOCIDADE)

\section{Análise de Sensibilidade dos Resultados}
\subsection{Caso 1 - Variação da frequência}
Este caso investiga como a variação da frequência natural afeta o modelo dinâmico do sistema. A frequência natural, sendo um parâmetro crítico, influencia diretamente o comportamento da planta do modelo, com implicações notáveis nas amplitudes dos desvios e na necessidade de compensação.

Observamos que sistemas com maior rigidez, caracterizados por frequências naturais mais altas, exibem menores amplitudes de desvio e uma reduzida necessidade de compensação. Esta tendência é claramente ilustrada ao comparar as Figuras \ref{fig:t_1a_vels}, \ref{fig:t_1b_vels} e \ref{fig:t_1c_vels}, que mostram o comportamento das velocidades em x e y para diferentes configurações de frequência.

\begin{figure}[H]
    \begin{center}
    \caption{Caso 1A - Comportamento no tempo das velocidades em x e y da ponta e da base}
    \includegraphics[scale=0.44]{Teste 1 A vels}
    \label{fig:t_1a_vels}
    \end{center}
\end{figure}

\begin{figure}[H]
    \begin{center}
    \caption{Caso 1B - Comportamento no tempo das velocidades em x e y da ponta e da base}
    \includegraphics[scale=0.44]{Teste 1 B vels}
    \label{fig:t_1b_vels}
    \end{center}
\end{figure}

\begin{figure}[H]
    \begin{center}
    \caption{Caso 1C - Comportamento no tempo das velocidades em x e y da ponta e da base}
    \includegraphics[scale=0.44]{Teste 1 C vels}
    \label{fig:t_1c_vels}
    \end{center}
\end{figure}

A análise é aprofundada ao examinar as Figuras \ref{fig:t_1a_des}, \ref{fig:t_1c_des}, \ref{fig:t_1a_pos} e \ref{fig:t_1c_pos}, que revelam as diferenças no comportamento do deslocamento e no caminho percorrido pela ponta do sistema nos extremos do caso (A e C). Estes resultados indicam que variações na frequência natural não apenas afetam as propriedades dinâmicas, mas também têm implicações diretas no controle e precisão do sistema.

\begin{figure}[H]
    \begin{center}
    \caption{Caso 1A - Comportamento no tempo dos deslocamentos em x e y da ponta e da base}
    \includegraphics[scale=0.44]{Teste 1 A des}
    \label{fig:t_1a_des}
    \end{center}
\end{figure}

\begin{figure}[H]
    \begin{center}
    \caption{Caso 1C - Comportamento no tempo dos deslocamentos em x e y da ponta e da base}
    \includegraphics[scale=0.44]{Teste 1 C des}
    \label{fig:t_1c_des}
    \end{center}
\end{figure}

\begin{figure}[H]
    \begin{center}
    \caption{Caso 1A - Caminho percorrido x vs y da ponta e da base}
    \includegraphics[scale=0.44]{Teste 1 A pos}
    \label{fig:t_1a_pos}
    \end{center}
\end{figure}

\begin{figure}[H]
    \begin{center}
    \caption{Caso 1C - Caminho percorrido x vs y da ponta e da base}
    \includegraphics[scale=0.44]{Teste 1 C pos}
    \label{fig:t_1c_pos}
    \end{center}
\end{figure}

Importante também é o impacto dessas variações na frequência natural no tempo de simulação. Observamos que os tempos de simulação variaram de 96,76 segundos no Caso A até 80,77 segundos no Caso C, com o Caso B intermediário, em 89,04 segundos. Esta tendência sugere que sistemas com maior rigidez (frequências naturais mais altas) não apenas melhoram a eficácia em termos de cumprimento das restrições, mas também podem levar a uma simulação mais rápida, apesar de todos os casos utilizarem vetores de posição com 241 elementos. Esta observação é crítica para aplicações práticas, onde a eficiência computacional e a rapidez na resposta são essenciais.

\subsection{Caso 2 - Variação do coeficiente de amortecimento}
Neste caso, investigamos o impacto de diferentes coeficientes de amortecimento no comportamento do sistema. As Figuras \ref{fig:t_2b_pos} e \ref{fig:t_2c_pos} são cruciais para esta análise, pois ilustram as trajetórias espaciais para as variações B e C do coeficiente de amortecimento. Estes resultados são contrastados com a Figura \ref{fig:t_padr_pos}, que representa um coeficiente intermediário, oferecendo uma comparação direta entre os extremos e o caso base.

\begin{figure}[H]
    \begin{center}
    \caption{Caso 2B - Caminho percorrido x vs y da ponta e da base}
    \includegraphics[scale=0.44]{Teste 2 B pos}
    \label{fig:t_2b_pos}
    \end{center}
\end{figure}

\begin{figure}[H]
    \begin{center}
    \caption{Caso 2C - Caminho percorrido x vs y da ponta e da base}
    \includegraphics[scale=0.44]{Teste 2 C pos}
    \label{fig:t_2c_pos}
    \end{center}
\end{figure}

Em termos de desempenho computacional, todos os casos mantiveram uma consistência, com 241 elementos nos vetores de posição. Os tempos de simulação foram de 69, 89 e 52 segundos para os Casos 2A, 2B e 2C, respectivamente. Estes resultados indicam que, apesar das variações no coeficiente de amortecimento, a simulação manteve uma eficiência computacional comparável entre os diferentes casos.

\subsection{Caso 3 - Variação na aceleração}
Neste caso, focamos na influência da variação da aceleração na geração de comandos do sistema. A Figura \ref{fig:t_3a_vels} é central nesta análise, pois ela revela uma forma distinta na curva de velocidade. Com a redução da aceleração, a velocidade não atingiu o patamar desejado, resultando em uma curva com formato triangular. Este comportamento destaca a sensibilidade do sistema às mudanças na aceleração e como isso afeta a capacidade de atingir velocidades alvo.

\begin{figure}[H]
    \begin{center}
    \caption{Caso 3A - Comportamento no tempo das velocidades em x e y da ponta e da base}
    \includegraphics[scale=0.44]{Teste 3 A vels}
    \label{fig:t_3a_vels}
    \end{center}
\end{figure}

Uma possível explicação para as dificuldades observadas no Caso A pode ser encontrada no tamanho dos vetores de posição utilizados. No Caso A, o vetor de posições tinha 401 elementos e a simulação levou 195 segundos para ser concluída. Em contraste, o Caso B utilizou um vetor de 221 elementos e completou a simulação em 78 segundos. Estes resultados indicam que ajustes na aceleração podem ter implicações significativas não apenas no desempenho do sistema, mas também na eficiência computacional da simulação.

\subsection{Caso 4 - Variação dos passos de tempo}
Este caso destaca como diferentes resoluções na malha de tempo afetam o comportamento do sistema. As Figuras \ref{fig:t_5a_vels}, \ref{fig:t_5b_vels} e \ref{fig:t_5c_vels} são fundamentais para entender a influência deste parâmetro nas oscilações observadas nos gráficos de velocidade. Cada uma destas figuras representa um cenário distinto, evidenciando como a resolução da malha de tempo pode alterar as dinâmicas de velocidade do sistema.

\begin{figure}[H]
    \begin{center}
    \caption{Caso 4A - Comportamento no tempo das velocidades em x e y da ponta e da base}
    \includegraphics[scale=0.44]{Teste 5 A vels}
    \label{fig:t_5a_vels}
    \end{center}
\end{figure}

\begin{figure}[H]
    \begin{center}
    \caption{Caso 4B - Comportamento no tempo das velocidades em x e y da ponta e da base}
    \includegraphics[scale=0.44]{Teste 5 B vels}
    \label{fig:t_5b_vels}
    \end{center}
\end{figure}

\begin{figure}[H]
    \begin{center}
    \caption{Caso 4C - Comportamento no tempo das velocidades em x e y da ponta e da base}
    \includegraphics[scale=0.44]{Teste 5 C vels}
    \label{fig:t_5c_vels}
    \end{center}
\end{figure}

Por outro lado, ao examinar os gráficos de deslocamento nas Figuras \ref{fig:t_5a_des}, \ref{fig:t_5b_des} e \ref{fig:t_5c_des}, notamos que as diferenças são menos pronunciadas. Isso sugere que, embora a resolução da malha de tempo tenha um impacto significativo na velocidade, seu efeito nos deslocamentos é mais sutil.

\begin{figure}[H]
    \begin{center}
    \caption{Caso 4A - Comportamento no tempo dos deslocamentos em x e y da ponta e da base}
    \includegraphics[scale=0.44]{Teste 5 A des}
    \label{fig:t_5a_des}
    \end{center}
\end{figure}

\begin{figure}[H]
    \begin{center}
    \caption{Caso 4B - Comportamento no tempo dos deslocamentos em x e y da ponta e da base}
    \includegraphics[scale=0.44]{Teste 5 B des}
    \label{fig:t_5b_des}
    \end{center}
\end{figure}

\begin{figure}[H]
    \begin{center}
    \caption{Caso 4C - Comportamento no tempo dos deslocamentos em x e y da ponta e da base}
    \includegraphics[scale=0.44]{Teste 5 C des}
    \label{fig:t_5c_des}
    \end{center}
\end{figure}

Podemos notar também o impacto na velocidade e facilidade de se convergir, podendo ser observado na Figura \ref{fig:t_5_viab}, assim
como os tempos de simulação $0,7 s$, $3 s$ e $279 s$ para os Casos A, B e C respectivamente. Além do tamanho dos vetores que está diretamente relacionado,
respectivamente em 25, 49 e 481 para os Casos A, B e C.
\subsection{Caso 5 - Variação da velocidade}
Este caso foca na análise dos impactos causados por diferentes velocidades desejadas, ajustadas através do Gcode de entrada. Investigamos como as velocidades mais baixas e mais altas influenciam a dinâmica do sistema, tanto em termos de desempenho quanto de eficiência computacional.

No Caso A, onde a velocidade desejada é mais baixa, notamos um impacto significativo no tamanho dos vetores e no tempo de simulação. Devido à menor velocidade, o sistema leva mais tempo para completar o percurso, resultando em uma malha mais densa de pontos, dada a resolução de \(dt\) constante. Isso se traduz em um tempo de simulação mais longo de 219 segundos e 421 elementos nos vetores de posição.

Em contraste, o Caso B, com uma velocidade desejada mais alta, apresentou um tempo de simulação substancialmente menor de 60 segundos e 181 elementos nos vetores de posição. Interessante notar que, apesar da maior velocidade desejada e da manutenção da mesma aceleração máxima na geração de comando, a curva de velocidade neste caso assemelha-se à observada no Caso 3A, como mostrado na Figura \ref{fig:t_4b_vels}. A correspondente curva de deslocamento é ilustrada na Figura \ref{fig:t_4b_des}, destacando as diferenças no comportamento do sistema sob essas condições.

\begin{figure}[H]
    \begin{center}
    \caption{Caso 5B - Comportamento no tempo das velocidades em x e y da ponta e da base}
    \includegraphics[scale=0.44]{Teste 4 B vels}
    \label{fig:t_4b_vels}
    \end{center}
\end{figure}

\begin{figure}[H]
    \begin{center}
    \caption{Caso 5B - Comportamento no tempo dos deslocamentos em x e y da ponta e da base}
    \includegraphics[scale=0.44]{Teste 4 B des}
    \label{fig:t_4b_des}
    \end{center}
\end{figure}

\section{Discussão Integrada dos Resultados}
Esta seção oferece uma análise integrada dos resultados obtidos nos diferentes casos de estudo, considerando as variáveis-chave e suas influências no comportamento do sistema simulado. A discussão foca em identificar padrões comuns, divergências e as implicações práticas dos resultados obtidos.

\subsection{Influência dos Parâmetros Variáveis}
A análise de sensibilidade realizada nos Casos 1 a 5 revelou que variações em parâmetros específicos, como frequência natural, coeficiente de amortecimento, aceleração, resolução de malha de tempo e velocidade desejada, têm impactos significativos e distintos no comportamento do sistema. 

No Caso 1, observou-se que a variação da frequência natural afeta diretamente a amplitude dos desvios e a necessidade de compensação. No Caso 3, a alteração da aceleração demonstrou como esse parâmetro pode influenciar a forma da curva de velocidade e a eficiência da simulação. O Caso 4 destacou a importância da resolução da malha de tempo, especialmente em relação às oscilações de velocidade. Por fim, o Caso 5 ilustrou como a velocidade desejada afeta tanto o tempo de simulação quanto o tamanho dos vetores de posição.

\subsection{Comportamento e Convergência do Sistema}
Em todos os casos, observou-se uma tendência de convergência para soluções viáveis, embora com variações na velocidade e facilidade desta convergência. 

% Isso sugere que, enquanto o sistema é robusto e adaptável a uma ampla gama de condições, existem limites críticos além dos quais a eficácia do sistema começa a diminuir.

\subsection{Considerações futuras}
Uma possível abordagem a ser explorada utilizando a ideia do método deste trabalho é a sobreposição de algoritmos, onde
um método referenciado em uma planta do sistema poderia buscar remover uma parcela das vibrações, atuando de forma estagiada,
com a participação de um método como \textit{InputShaping} para atacar as vibrações remanescentes.

Uma possibilidade que o tipo de método abordado neste trabalho oferece é a capacidade de otimizar os parâmetros da planta para uma determinada posição.
Assim, oferecendo a capacidade de se ajustar em grande nível de detalhe as peculiaridades do sistema, podendo até
construir a malha utilizando sensores, semelhantemente a rotinas de configuração de \textit{InputShaping} que amostram
o comportamento em frequência no ponto central da impressora. Considerando também que a utilização desse tipo de malha,
teria pouco impacto computacional.


