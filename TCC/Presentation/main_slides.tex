\documentclass[aspectratio=169]{beamer}
\usepackage{tikz}
\newcommand{\backgroundlogo}{
  \begin{tikzpicture}[remember picture,overlay]
    \node[opacity=0.1,anchor=center] at (current page.center) {%
      \includegraphics[width=10cm]{logoufla.jpg}
    };
  \end{tikzpicture}
}

\usetheme{default}

\definecolor{UFLAblue}{HTML}{224271}

\definecolor{UFLAgreen}{HTML}{00793c}

\setbeamercolor{title}{fg=UFLAblue}
\setbeamercolor{frametitle}{fg=UFLAblue}

\setbeamercolor{titlelike}{fg=UFLAblue}
\setbeamercolor{section in toc}{fg=UFLAblue}

\setbeamercolor{secondary}{fg=black}

\setbeamercolor{normal text}{fg=black}

\setbeamertemplate{section page}
{
    \begin{centering}
    \begin{beamercolorbox}[sep=12pt,center]{part title}
    \usebeamerfont{section title}\insertsection\par
    \end{beamercolorbox}
    \end{centering}
}

\AtBeginSection{\frame{\sectionpage}}

\title[Presente Trabalho]{\textbf{CONTROLE DE TRAJETÓRIA DE IMPRESSORAS 3D UTILIZANDO ALGORITMO ITERATIVO E PROGRAMAÇÃO NÃO LINEAR}
}
\author[Joao]{João Vivas Cisalpino}

\institute[UFLA]{Orientador: Wander Gustavo Rocha Vieira \\ Universidade Federal de Lavras}

\date{08/12/2023} % Data da apresentação

\begin{document}

\begin{frame}
  \backgroundlogo
  \titlepage
\end{frame}

\logo{\includegraphics[width=2cm]{logoufla.jpg}}

\begin{frame}
  \frametitle{Sumário}
  \setcounter{tocdepth}{1}
  \tableofcontents
  % Conteúdo do slide aqui
\end{frame}

\section{\insertsectionnumber . Introdução}

\begin{frame}
  \frametitle{\insertsection}
  % Conteúdo do slide aqui
\end{frame}


\subsection{\insertsectionnumber .\insertsubsectionnumber . Objetivos}
\begin{frame}
  \frametitle{\insertsubsection}
  % Conteúdo do slide aqui
\end{frame}

\section{\insertsectionnumber . Referencial Teórico}
\subsection{\insertsectionnumber .\insertsubsectionnumber . Manufatura Aditiva}
\begin{frame}
  \frametitle{\insertsubsection}
  % Conteúdo do slide aqui
\end{frame}

\subsection{\insertsectionnumber .\insertsubsectionnumber . \textit{Fused Deposition Modeling} (FDM)}
\begin{frame}
  \frametitle{\insertsubsection}
  % Conteúdo do slide aqui
\end{frame}

\subsection{\insertsectionnumber .\insertsubsectionnumber . Geração de Comando}
\begin{frame}
  \frametitle{\insertsubsection}
  % Conteúdo do slide aqui
\end{frame}

\subsection{\insertsectionnumber .\insertsubsectionnumber . Geração de Trajetória}
\begin{frame}
  \frametitle{\insertsubsection}
  % Conteúdo do slide aqui
\end{frame}

\subsection{\insertsectionnumber .\insertsubsectionnumber . Técnicas de Controle}
\begin{frame}
  \frametitle{\insertsubsection}
  % Conteúdo do slide aqui
\end{frame}

\section{\insertsectionnumber . Metodologia}

\subsection{\insertsectionnumber .\insertsubsectionnumber . Geração de Trajetória}
\begin{frame}
  \frametitle{\insertsubsection}
  % Conteúdo do slide aqui
\end{frame}

\subsection{\insertsectionnumber .\insertsubsectionnumber . Modelagem Dinâmica de uma Impressora 3D}
\begin{frame}
  \frametitle{\insertsubsection}
  % Conteúdo do slide aqui
\end{frame}

\subsection{\insertsectionnumber .\insertsubsectionnumber . Controle de Trajetória}
\begin{frame}
  \frametitle{\insertsubsection}
  % Conteúdo do slide aqui
\end{frame}

\section{\insertsectionnumber . Resultados}

\subsection{\insertsectionnumber .\insertsubsectionnumber . Simulação Referência}
\begin{frame}
  \frametitle{\insertsubsection}
  % Conteúdo do slide aqui
\end{frame}

\subsection{\insertsectionnumber .\insertsubsectionnumber . Simulação com Parâmetros Variados}
\begin{frame}
  \frametitle{\insertsubsection}
  % Conteúdo do slide aqui
\end{frame}

\subsection{\insertsectionnumber .\insertsubsectionnumber . Aplicação do Método Runge-Kutta}
\begin{frame}
  \frametitle{\insertsubsection}
  % Conteúdo do slide aqui
\end{frame}

\subsection{\insertsectionnumber .\insertsubsectionnumber . Resultado da Simulação - Variação da Frequência Natural}
\begin{frame}
  \frametitle{\insertsubsection}
  % Conteúdo do slide aqui
\end{frame}

\subsection{\insertsectionnumber .\insertsubsectionnumber . Resultado da Simulação - Variação do Coeficiente de Amortecimento}
\begin{frame}
  \frametitle{\insertsubsection}
  % Conteúdo do slide aqui
\end{frame}

\subsection{\insertsectionnumber .\insertsubsectionnumber . Resultado da Simulação - Variação na Aceleração}
\begin{frame}
  \frametitle{\insertsubsection}
  % Conteúdo do slide aqui
\end{frame}

\subsection{\insertsectionnumber .\insertsubsectionnumber . Resultado da Simulação - Variação da Velocidade desejada}
\begin{frame}
  \frametitle{\insertsubsection}
  % Conteúdo do slide aqui
\end{frame}

\subsection{\insertsectionnumber .\insertsubsectionnumber . Resultado da Simulação - Variação do Passo de tempo}
\begin{frame}
  \frametitle{\insertsubsection}
  % Conteúdo do slide aqui
\end{frame}

\section{\insertsectionnumber . Conclusão}

\begin{frame}
  \frametitle{\insertsection}
  % Conteúdo do slide aqui
\end{frame}

% Remove a logo para o próximo slide
\setbeamertemplate{logo}{}
\begin{frame}
  
\backgroundlogo
\begin{center}
  \huge{Obrigado!}
\end{center}
\end{frame}

\end{document}